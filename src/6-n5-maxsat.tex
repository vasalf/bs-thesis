\newgeometry{top=2cm,bottom=2cm,left=3cm,right=1.5cm,nohead,includeheadfoot}

\section{Разбор 5-переменных}
\label{sec:n5}

\subsection{Общие наблюдения}
\label{subsec:n5:observations}

\firstpar{}Итак, после того, как правило расщепления \ref{brule:measure:sixplus} неприменимо, остался экземпляр задачи $(n,5)$-MAXSAT. В данном разделе представлен разбор случаев, в совокупности позволяющих избавиться от 5-переменных. Прежде всего, несколько наблюдений.

Во-первых, не умаляя общности, 5-переменная может быть $(4,1)$-переменной или $(3,2)$-переменной. В силу правила упрощения \ref{rrule:common:unit-clauses}, в первом случае переменная может встречаться в дизъюнктах длины 1 лишь отрицательно, во втором случае -- либо дважды отрицательно, либо однажды положительно или отрицательно. Каждый из этих случаев разобран в разделах этой главы.

Во-вторых, вспомним доказательство леммы \ref{lemma:measure:branch-on}. В доказательстве изменение уменьшенной длины формулы считалось как сумма изменений весов всех переменных. Для 5-переменных оказывается, что случаев переменных-соседей немного. Все эти случаи приведены в таблице \ref{table:n5:varcases}. В первом столбце перечислено количество вхождений во всю формулу. Во втором столбце перечислено изменение длины до применения правил упрощения, то есть количество литералов этой переменной, которые могут одновременно иметь соседями один и тот же литерал переменной, по которой мы пытаемся расщепиться. В третьем столбце приведено изменение веса этой переменной в уменьшенной длине в таком случае.

\begin{table}[ht]
 \centering
 \caption{Случаи переменных-соседей для $(n,5)$-MAXSAT}
 \begin{tabular}{|c|c|c|}
  \hline
  Тип переменной & $\Delta L$ & $\Delta d$ \\
  \hline\hline
  \multirow{2}{*}{3-переменная}
                 & 1          & 2 \\
                 & 2          & 2 \\
  \hline
  \multirow{4}{*}{4-переменная}
                 & 1          & 2 \\
                 & 2          & 4 \\
                 & 3          & 4 \\
                 & 4          & 4 \\
  \hline
  \multirow{4}{*}{5-переменная}
                 & 1          & 1 \\
                 & 2          & 3 \\
                 & 3          & 5 \\
                 & 4          & 5 \\
  \hline
 \end{tabular}
 \label{table:n5:varcases}
\end{table}

В таблице отсутствуют 3-переменные, встречающиеся среди соседей такой переменной трижды, поскольку, как указано в доказательстве леммы \ref{lemma:measure:branch-on}, правило упрощения \ref{rrule:common:xu19rr9} позволяет сократить такие случаи. Также в таблице не указаны 5-переменные, встречающиеся пятикратно, так как пять литералов такой переменной должны были бы иметь соседями один и тот же литерал, что невозможно, поскольку в силу правила упрощения \ref{rrule:common:i0} в $(n,5)$-MAXSAT формуле каждый литерал встречается не более чем четырежды.

Лемма \ref{lemma:measure:branch-on} фактически утверждает, что $\Delta L \leq \Delta d$. Для $(n,5)$-MAXSAT можно заметить к тому же, что почти во всех строках $\Delta L < \Delta d$. Равенство достигается лишь для 3-переменных, встречающихся дважды, 4-переменных, встречающихся четырежды, или 5-переменных, встречающихся однажды. Более того, второй из этих трёх случаев встречается только при разборе $(4,1)$-переменных, но не при разборе $(3,2)$-переменных.

Таким образом, верно неформальное усиление леммы \ref{lemma:measure:branch-on}: если среди соседей есть переменные, не являющиеся одним из этих трёх (а после разбора $(4,1)$-переменных двух) случаев, то уменьшенная мера уменьшается даже больше, чем длина. Это позволяет строить эффективные правила расщепления для 5-переменных.

Такое утверждение позволяет разобрать огромное количество крайних случаев. В общем же случае может оказаться, что все соседи какой-то переменной -- 5-переменные, встречающиеся там однажды. В таком случае можно этим воспользоваться и вывести правило расщепления, означивающее одновременно многих соседей. Это правило формально закреплено в двух следующих леммах.

\begin{lemma}
 Пусть $x$ -- $(3,2)$-переменная следующего вида:

 $$
  (x \vee C_1) \wedge (x \vee C_2) \wedge (x \vee C_3) \wedge (\ovl{x} \vee D_1) \wedge (\ovl{x} \vee D_2) \wedge F'
 $$

 Пусть при этом $|C_1| = 1$, а $C_2$ и $C_3$ непусты. Тогда, если существует означивание переменных, одновременно выполняющее $C_2$ и $C_3$ и не выполняющее $C_1$, $D_1$ и $D_2$, расщепление на следующие три случая корректно:

 \begin{enumerate}
  \item $x = 1$
  \item $x = 0$, $C_1 = 1$
  \item $x = 0$, $C_1 = 0$, $C_2 = C_3 = 1$, и, если для какого-то $i$ дизъюнкт $D_i$ непуст, $D_i = 0$
 \end{enumerate}

 \label{lemma:n5:3-cases}
\end{lemma}

\begin{proof}
 Во-первых, докажем, что существует оптимальное означивание, в котором или $x$ равен единице, или два из трёх дизъюнктов $C_i$ выполнены.

 Рассмотрим какое-то оптимальное означивание. Пусть в нём $x = 0$ и из $C_i$ выполнено не более одного дизъюнкта. Тогда всего из дизъюнктов, в которых встречается $x$, выполнено не более трёх. Означив в таком случае $x = 1$, мы из этих пяти дизъюнктов выполним хотя бы столько же, и не изменим значения других дизъюнктов.

 Более того, если в оптимальном означивании хотя бы один из $D_i$ равен единице, то утверждение усиливается до ``все $C_i$ должны быть выполнены'' в силу того, что означивание $x = 1$ выполняет хотя бы четыре из пяти дизъюнктов.

 Таким образом, существует оптимальное означивание, в котором выполняется одно из трёх утверждений: либо $x = 1$, либо $x = 0$ и $C_1 = 1$, либо, если $x = 0$ и $C_1 = 0$, то $C_2$ и $C_3$ равны единице. Более того, поскольку в последнем случае выполняются два из трёх $C_i$, в таком случае обязательно все непустые $D_i$ равны нулю.
\end{proof}

\begin{lemma}
 Пусть $x$ -- $(3,2)$-переменная такого же вида, как в лемме \ref{lemma:n5:3-cases}, где $|C_1| = 1$, а $C_2$ и $C_3$ непусты. Пусть при этом не существует означивания переменных, одновременно выполняющего $C_2$ и $C_3$ и не выполняющего $C_1$, $D_1$ и $D_2$. Тогда расщепление на следующие три случая корректно:

 \begin{enumerate}
  \item $x = 1$
  \item $x = 0$, $C_1 = 1$
 \end{enumerate}
 \label{lemma:n5:2-cases}
\end{lemma}

\begin{proof}
 В лемме \ref{lemma:n5:3-cases} доказано, что существует оптимальное означивание переменных, удовлетворяющее одному из трёх случаев. В данной лемме дополнительно в условиях указана невозможность существования означивания, удовлетворяющего третьему из них. Значит, остаются первые два.
\end{proof}

Как показано ниже, двух правил расщепления из лемм \ref{lemma:n5:3-cases} и \ref{lemma:n5:2-cases} в совокупности с расщеплением по переменным и несколькими правилами упрощения оказывается достаточно.

\subsection{Разбор $(4,1)$-переменных}
\label{subsec:n5:41}

\firstpar{}В первую очередь докажем общие правила упрощения и расщепления для $(i,1)$-переменных.

\begin{rrule}
 Пусть $x$ -- $(i,1)$-переменная ($i \geq 2$) в формуле вида

 $$
  (x \vee C_1) \wedge \ldots \wedge (x \vee C_i) \wedge (\ovl{x} \vee D) \wedge F'
 $$

 Пусть для какого-то $j$ выполняется $|C_j| = 1$, и литерал оттуда присутствует в $D$. Тогда можно назначить $x = 1$.
 \label{rrule:n5:i1}
\end{rrule}

\begin{proof}
 Обозначим $C_j = l$.

 Если в оптимальном назначении $l = 1$, то $D$ выполнен, и, таким образом, поскольку единственный дизъюнкт с $\ovl{x}$ выполнен, назначение $x = 1$ выполняет не меньше дизъюнктов в таком означивании, чем $x = 0$.

 Если же $l = 0$, и при этом $x = 0$, то из дизъюнктов с переменной $x$ выполнено в таком означивании не более $i$, в то время как назначение в этом означивании $x = 1$ выполнит хотя бы $i$.
\end{proof}

\begin{lemma}
 Пусть $x$ -- $(i,1)$-переменная ($i \geq 2$) в формуле вида

 $$
  (x \vee C_1) \wedge \ldots \wedge (x \vee C_i) \wedge (\ovl{x} \vee D) \wedge F'
 $$

 Тогда расщепление на следующие два случая корректно:
 \begin{enumerate}
  \item $x = 1$
  \item $x = 0$, если $D$ непусто, $D = 0$, и, если для какого-то $j$ выполняется $|C_j| = 1$, то $C_j = 1$
 \end{enumerate}
 \label{lemma:n5:i1}
\end{lemma}

\begin{proof}
 Покажем, что существует оптимальное означивание, в котором либо $x = 1$, либо $x = 0$ и тогда все $C_j$ выполнены, а $D$ не выполнено.

 Рассмотрим какое-то оптимальное означивание. Пусть в нём $x = 0$, но один из $C_j$ не выполнен. Тогда из всех дизъюнктов, содержащих $x$, выполнены не более $i$. Тогда при означивании $x = 1$ из этих дизъюнктов будет выполнено хотя бы $i$, а значения других дизъюнктов не изменяются. Таким образом, мы получили оптимальное означивание, в котором $x = 1$.

 Теперь пусть в оптимальном означивании $x = 0$ и все $C_j$ выполнены, но также выполнен $D$. Тогда при означивании $x = 1$ выполнимость ни одного дизъюнкта не изменится. Таким образом, мы снова получили оптимальное означивание, в котором $x = 1$.
\end{proof}

\begin{note}
 Отметим, что такое назначение всегда выполнимо. А именно, если два $C_j$ имеют длину 1 и содержат противоположные дизъюнкты одной и той же переменной, применимо правило упрощения \ref{rrule:common:almost-common}. Также, если $C_j$ имеет длину 1, то этот литерал не может содержаться в $D$ в силу правила \ref{rrule:n5:i1}.
\end{note}

Мы обозначим $(4,1)$-переменную $x$ также, как в формулировке леммы \ref{lemma:n5:i1}:

$$
 (x \vee C_1) \wedge (x \vee C_2) \wedge (x \vee C_3) \wedge (x \vee C_4)
$$

Здесь, как указано выше, все $C_i$ непусты, а $D$ может быть пустым.

Воспользуемся только что доказанной леммой.

\begin{brule}
 Если $x$ -- $(4,1)$-переменная, расщепиться по правилу из леммы \ref{lemma:n5:i1}.

 Это даёт как минимум $(7,9)$-расщепление.
 \label{brule:n5:41}
\end{brule}

\begin{proof}
 Во-первых, если $|D| > 0$, это хотя бы $(7,9)$-расщепление. В первом случае длина уменьшается хотя бы на 9 (пять литералов $x$ и как минимум четыре литерала в $C_j$), а значит, по лемме \ref{lemma:measure:branch-on}, уменьшенная длина уменьшается хотя бы на 9. Во втором случае мы убираем 5-переменную $x$ и $3^+$-переменную из $D$, таким образом уменьшая меру $d$ как минимум на 7.

 Во-вторых, если $|D| = 0$, но существует такое $j$, что $|C_j| = 1$, то это точно также хотя бы $(7,9)$-расщепление: во втором случае мы убираем переменную не из $D$, а из $C_j$, а в остальном доказательство повторяет предыдущий абзац.

 Наконец, если $|D| = 0$ и все $|C_j| > 1$, то $\sum_j |C_j| \geq 8$. Тогда по лемме \ref{lemma:measure:branch-on} это хотя бы $(5,13)$-расщепление.

 Так как вектор $(7,9)$ хуже, он и является оценкой для худшего случая.
\end{proof}

После того, как указанные правила неприменимы, в формуле не осталось $(4,1)$-переменных.

\subsection{Разбор $(3,2)$-переменных в двух дизъюнктах длины 1}
\label{subsec:n5:32-3uc}

\firstpar{}$(3,2)$-переменные могут входить в дизъюнкты длины один дважды лишь отрицательно в силу правила упрощения \ref{rrule:common:i0}. Таким образом, в данном подразделе рассматривается следующий случай:

$$
 (x \vee C_1) \wedge (x \vee C_2) \wedge (x \vee C_3) \wedge \ovl{x} \wedge \ovl{x}
$$

Основное наблюдение заключается в правиле расщепления \ref{brule:n5:32-2uc}, но для применимости этого правила требуется ввести несколько правил упрощения.

\begin{rrule}
 Если в обозначениях выше в дизъюнктах $C_i$ содержатся противоположные дизъюнкты, можно назначить $x = 0$.
 \label{rrule:n5:32-2uc:compl}
\end{rrule}

\begin{proof}
 В таком случае в любом означивании хотя бы один из дизъюнктов $C_i$ выполнен. Значит, назначение $x = 0$ в оптимальном означивании выполняет хотя бы три дизъюнкта из дизъюнктов с переменной $x$, в то время как назначение $x = 1$ выполняет ровно три.
\end{proof}

\begin{rrule}
 Если в обозначениях выше объединение всех $C_i$ содержит литералы лишь одной переменной (и, так как правило \ref{rrule:n5:32-2uc:compl} неприменимо, все эти литералы совпадают с каким-то литералом $l$), можно означить $x = 0$ и $l = 1$.
 \label{rrule:n5:32-2uc:same}
\end{rrule}

\begin{proof}
 После того, как правило упрощения \ref{rrule:common:xu19rr9} неприменимо и в формуле не осталось $(4,1)$-переменных, $l$ обязан быть литералом $(3,2)$-переменной $y$.

 Рассмотрим оптимальное означивание переменных. Пусть в нём $x = 1$. Тогда из дизъюнктов с $x$ и $y$ выполнено не более пяти (а на самом деле, ровно пять, так как вне $C_i$ не встречается литерал $l$, но лишь $\ovl{l}$, и таким образом, дизъюнкты с $\ovl{l}$ обязательно выполнены). При означивании же $x = 0$ и $l = 1$ выполнено не менее пяти. Таким образом, такое означивание не уменьшает количество выполненных дизъюнктов.
\end{proof}

Теперь можно перейти к правилу расщепления для таких переменных.

\begin{brule}
 Если $x$ -- $(3,2)$-переменная, входящая в два дизъюнкта длины 1, в обозначениях выше, расщепиться на два случая:
 \begin{enumerate}
  \item $x = 0$
  \item $x = 1$ и $C_1 = C_2 = C_3 = 0$.
 \end{enumerate}

 Это даёт как минимум $(5,12)$-расщепление.
 \label{brule:n5:32-2uc}
\end{brule}

\begin{proof}
 Корректность правила содержит в себе ту же идею, что и в правиле упрощения \ref{rrule:n5:32-2uc:compl}. Если в оптимальном означивании хотя бы один из дизъюнктов $C_i$ выполнен, то при назначении $x = 1$ выполняется ровно три дизъюнкта с переменной $x$, а при назначении $x = 0$ выполняется не менее трёх из этих дизъюнктов. Значит, существует оптимальное означивание переменных, в котором или $x = 0$, или $x = 1$ и все $C_i$ равны нулю.

 В силу правила упрощения \ref{rrule:n5:32-2uc:compl} такое означивание всегда сделать возможно.

 Перейдём к анализу вектора расщепления.

 Если в $C_j$ содержатся литералы хотя бы четырёх различных переменных, то в первом случае уменьшенная длина уменьшается на 5 (убирается 5-переменная $x$), а во втором хотя бы на 13: исчезает 5-переменная $x$ и четыре $3^+$-переменных из $C_i$. Таким образом, в этом случае это хотя бы $(5,13)$-расщепление.

  Если в $C_j$ содержатся литералы хотя бы трёх различных переменных, и хотя бы одна из этих переменных является $4^+$-переменных, это точно также хотя бы $(5,13)$-расщепление: во втором случае исчезнут две $3^+$-переменных и $4^+$-переменная с таким же суммарным вкладом.

  Если в $C_j$ содержатся литералы ровно трёх различных переменных, то в силу правил упрощения \ref{rrule:common:xu19rr9} и \ref{rrule:n5:32-2uc:compl} каждая такая переменная обязана входить в $C_j$ ровно однажды. Если хотя бы одно из этих вхождений является $(2,1)$-литералом, то в первом случае такой литерал окажется после применения правила расщепления в дизъюнкте длины 1, и по правилу упрощения \ref{rrule:common:unit-clauses} вся переменная будет элиминирована. Таким образом, в таком случае в первом случае мера $d$ уменьшится не меньше, чем на 7, а во втором не меньше, чем на 11 (исчезнет 5-переменная $x$ и три 3-переменные). Таким образом, тогда это не хуже, чем $(7,11)$-расщепление.

  Если же все эти литералы -- $(1,2)$-литералы, поскольку правило \ref{rrule:common:closed} неприменимо, у одной их этих переменных должен быть сосед, не являющийся ни $x$, ни соседом $x$. Тогда во втором случае хотя бы один литерал такого соседа элиминируется, и тогда это хотя бы $(5,12)$-расщепление: во втором случае исчезает 5-переменная $x$, три 3-переменных, и у одной из других переменных вес уменьшается ещё хотя бы на 1.

  Если в $C_j$ содержатся литералы ровно двух переменных, то в силу сказанного выше не может оказаться так, что обе этих переменных -- 3-переменные (так как 3-переменная может входить в $C_j$ лишь однажды). Тогда одна из этих переменных -- $4^+$-переменная, а другая -- $3^+$-переменная. Если при этом первая является $5$-переменной или вторая является $4^+$-переменной, то это хотя бы $(5,12)$-расщепление: во втором случае исчезает 5-переменная $x$ и две других переменных, суммарный вес которых хотя бы 7.

  Наконец, если в $C_j$ содержатся литералы ровно двух переменных, и это 3-переменная и 4-переменная, то среди $C_j$ должны быть два дизъюнкта, содержащие ровно по одному литералу, и это литерал 4-переменной. Тогда в первом случае после применения правила расщепления у этой 4-переменной останется два вхождения в дизъюнкт длины 1 и она будет элиминирована по правилу упрощения \ref{rrule:common:unit-clauses}. Таким образом, в первом случае мера $d$ уменьшится хотя бы на 9. Во втором случае исчезают как минимум 5-переменная, 4-переменная и 3-переменная, таким образом, это хотя бы $(9,11)$-расщепление.

  Худшим из представленных векторов является вектор $(5,12)$, таким образом, он и является оценкой для худшего случая.
\end{proof}

Это правило завершает раздел про $(3,2)$-переменные с двумя вхождениями в дизъюнкты длины 1.

\subsection{Разбор $(3,2)$-литералов в дизъюнкте длины 1}
\label{subsec:n5:32-+uc}

В данном разделе обсуждается случай, когда $(3,2)$-переменная положительно входит в дизъюнкт длины 1. Как обсуждалось выше, такая переменная не может входить более чем в один дизъюнкт длины 1. Таким образом, разбирается следующий случай.

$$
 x \wedge (x \vee C_1) \wedge (x \vee C_2) \wedge (\ovl{x} \vee D_1) \wedge (\ovl{x} \vee D_2)
$$

Продолжая практику отдельного вывода правил расщепления для крайних случаев и правил упрощения для случаев, когда такие правила неприменимы, предоставим сначала правило упрощения, а затем правило расщепления.

\begin{rrule}
 Если в обозначениях выше в дизъюнктах $D_i$ содержатся противоположные дизъюнкты, можно назначить $x = 1$.
 \label{rrule:n5:32-+uc}
\end{rrule}

\begin{proof}
  В таком случае хотя бы один из двух дизъюнктов $D_1$ и $D_2$ всегда верен.

  Рассмотрим оптимальное назначение переменных и пусть там $x = 0$. В таком случае из пяти дизъюнктов с переменной $x$ выполнено не более четырёх. В силу того, что из $D_1$ и $D_2$ есть хотя бы один выполненный, при назначении $x = 1$ в таком означивании выполнится хотя бы четыре из этих дизъюнктов, а выполненность других дизъюнктов не изменится. Таким образом, существует оптимальное назначение переменных с $x = 1$.
\end{proof}

\begin{brule}
 Если $x$ -- $(3,2)$-переменная, положительно входящая в дизъюнкт длины 1, в обозначениях выше, расщепиться на два случая:

 \begin{enumerate}
  \item $x = 1$
  \item $x = 0$, $D_1 = D_2 = 0$
 \end{enumerate}

 Это даёт хотя бы $(7,9)$-расщепление.
 \label{brule:n5:32-+uc}
\end{brule}

\begin{proof}
 Докажем корректность.

 Рассмотрим оптимальное назначение переменных и пусть там $x = 0$, но при этом какой-то из $D_1$ и $D_2$ равен единице. Но тогда точно так же, как в доказательстве правила упрощения \ref{rrule:n5:32-+uc}, выйдет, что назначение $x = 1$ в таком означивании выполняет не меньше дизъюнктов. Таким образом, существует оптимальное означивание, где либо $x = 1$, либо $x = 0$ и $D_1 = D_2 = 0$.

 В силу правила \ref{rrule:n5:32-+uc} всегда возможно назначить одновременно $D_1 = 0$ и $D_2 = 0$.

 Теперь докажем вектор расщепления.

 В первом случае, по лемме \ref{lemma:measure:branch-on}, мера $d$ уменьшается как минимум на 7 (так как длина уменьшается на 7: уничтожается 5-переменная $x$ и два литерала-соседа из $C_1$ и $C_2$).

 Во втором случае, если в $D_i$ встречаются хотя бы две различные переменные, мера $d$ уменьшается хотя бы на 9: убирается 5-переменная $x$ и две $3^+$-переменных.

 Если же там всего одна переменная, то в силу правил упрощения \ref{rrule:common:xu19rr9} и \ref{rrule:n5:32-+uc} это $4^+$-переменная. Тогда мера $d$ уменьшается также хотя бы на 9: убирается 5-переменная $x$ и $4^+$-переменная.
\end{proof}

Это завершает разбор случая положительного $(3,2)$-литерала в дизъюнкте длины 1.

\subsection{Разбор других $(3,2)$-переменных}
\label{subsec:n5:32-rest}

Теперь остались лишь $(3,2)$-переменные, входящие в дизъюнкты длины 1 не более чем однажды и только отрицательно. Этот случай обозначается так:

$$
 (x \vee C_1) \wedge (x \vee C_2) \wedge (x \vee C_3) \wedge (\ovl{x} \vee D_1) \wedge (\ovl{x} \vee D_2)
$$

Без ограничения общности будем считать, что $|D_1| \geq |D_2|$. Тогда из пяти дизъюнктов $C_i$ и $D_i$ лишь $D_2$ может быть пустым.

Прежде всего, общее правило расщепления, которое укажет дальнейшее направление разбора случаев.

\begin{brule}
 Пусть $x$ -- $(3,2)$-переменная в обозначениях выше. Если $\sum |C_i| + \sum |D_i| \geq 6$, расщепиться по $x$.

 Это даёт хотя бы $(6,10)$-расщепление.
 \label{brule:n5:32-rest:length}
\end{brule}

\begin{proof}
 Применим лемму \ref{lemma:measure:branch-on}. Поскольку $C_i$ и $D_1$ непусты, $\sum |C_i| \geq 3$ и $\sum |D_i| \geq 1$. Поскольку сумма этих двух чисел по условию не меньше шести, какое-то из них или оба должны быть больше.

 Если оба числа больше, это хотя бы $(7,9)$-расщепление.

 Если только первое число больше, оно больше хотя бы на 2, и тогда это хотя бы $(6,10)$-расщепление.

 Если только второе число больше, оно больше хотя бы на 2, и тогда это хотя бы $(8,8)$-расщепление.

 Худшим из трёх векторов является $(6,10)$, таким образом, он и является оценкой на худший случай.
\end{proof}

В частности, правило \ref{brule:n5:32-rest:length} применимо, если $\sum |C_i| \geq 5$. Таким образом, сейчас $3 \leq \sum |C_i| \leq 4$. Мы будем рассматривать случаи, когда эта сумма равна 3 и 4 отдельно, но перед этим выведем несколько общих для этих случаев правил расщепления.
Эти правила используют лемму \ref{lemma:n5:2-cases} для разбора случаев с малым количеством переменных-соседей у $x$, подготавливая почву для использования леммы \ref{lemma:n5:3-cases} в более общих случаях.

\begin{brule}
 Пусть $x$ -- $(3,2)$-переменная в обозначениях выше. Если существует такое $j$, что $|C_j| = 1$ и литерал из $C_j$ присутствует в хотя бы одном из $D_i$, расщепиться на два случая:

 \begin{enumerate}
  \item $x = 1$
  \item $x = 0$, $C_k = 1$, где $k$ такое, что $|C_k| = 1$ и $k \neq j$.
 \end{enumerate}

 Это даёт хотя бы $(8,8)$-расщепление.
 \label{brule:n5:32-rest:same}
\end{brule}

\begin{proof}
 Поскольку в таких условиях не может быть одновременно $C_j = 1$ и все $D_i = 0$, корректность этого правила следует из леммы \ref{lemma:n5:2-cases}.

 Отметим, что в таком случае $\sum |D_i| \geq 2$, так как если в объединении $D_j$ содержался бы единственный дизъюнкт, было бы применимо правило упрощения \ref{rrule:common:almost-common}. Более того, по той же причине в объединении $D_j$ если литерал переменной, отличной от $x$ и переменной из $C_j$.

 В первом случае по лемме \ref{lemma:measure:branch-on} мера $d$ уменьшается хотя бы на 8.

 Во втором случае элиминируется 5-переменная $x$, $3^+$-переменная из $C_j$ и один литерал ещё одной переменной. Таким образом, мера $d$ уменьшается хотя бы на 8.
\end{proof}

\begin{brule}
 Пусть $x$ -- $(3,2)$-переменная в обозначениях выше. Пусть $j$ такое, что $|C_j| = 1$. Обозначим тогда $C_j = l$. Если $\ovl{l}$ встречается в объединении $C_j$, расщепиться на два случая:

 \begin{enumerate}
  \item $x = 1$
  \item $x = 0$, $l = 1$
 \end{enumerate}
 
 Это даёт хотя бы $(7,9)$-расщепление.
 \label{brule:n5:32-rest:compl}
\end{brule}

\begin{proof}
 Поскольку в таких условиях не может быть одновременно $l = 1$ и все $D_i = 0$, корректность этого правила следует из леммы \ref{lemma:n5:2-cases}.

 Если $l$ -- литерал $4^+$-переменной, В первом случае по лемме \ref{lemma:measure:branch-on} тогда мера уменьшается хотя бы на 8. Во втором случае тогда уничтожается 5-переменная $x$ и $4^+$-переменная, таким образом, мера уменьшается хотя бы на 9.

 Если $l$ -- литерал 3-переменной, встречающейся в объединении $C_i$ дважды, это по правилу упрощения \ref{rrule:common:xu19rr9} должны быть вхождения разного знака, а так как правило упрощения \ref{rrule:common:almost-common} неприменимо, тогда длина дизъюнкта со вторым вхождением должна быть хотя бы 3. Тогда в первом случае по лемме \ref{lemma:measure:branch-on} мера уменьшается хотя бы на 9, а во втором уничтожаются 5-переменная $x$ и 3-переменная и таким образом мера уменьшается хотя бы на 7.

 Если $l$ -- литерал 3-переменной, встречающейся в объединении $C_i$ лишь однажды, в первом случае тогда эта 3-переменная полностью уничтожится. Тогда вес этой переменной в первом случае уменьшится больше, чем количество её литералов -- соседей $x$, и в первом случае мера уменьшится хотя бы на 9. Во втором же случае мера аналогично уменьшится хотя бы на 7.

 Поскольку из двух векторов худшим является $(7,9)$, он и является оценкой для худшего случая.
\end{proof}

\begin{brule}
 Пусть $x$ -- $(3,2)$-переменная в обозначениях выше. Если существуют $i$ и $j$ такие, что $|C_i| = |C_j| = 1$ и $C_i = C_j$, то расщепиться на два случая:

 \begin{enumerate}
  \item $x = 1$
  \item $x = 0$, $C_i = 1$
 \end{enumerate}

 Это даёт хотя бы $(8,9)$-расщепление.
 \label{brule:n5:32-rest:same-c}
\end{brule}

\begin{proof}
 Поскольку в таких условиях не может быть одновременно $C_i = 1$ и все $C_j = 0$, корректность этого правила следует из леммы \ref{lemma:n5:2-cases}.

 В силу правила упрощения \ref{rrule:common:xu19rr9} в таком случае литерал из $C_i$ и $C_j$ обязан быть литералом $4^+$-переменной.

 В первом случае по лемме \ref{lemma:measure:branch-on} мера уменьшается хотя бы на 8. Во втором случае означиваются 5-переменная $x$ и $4^+$-переменная, таким образом, мера уменьшается хотя бы на 9.
\end{proof}

Отметим, что такой же случай для противоположных литералов $C_i$ и $C_j$ решается правилом упрощения \ref{rrule:common:almost-common}.

Перейдём к разбору случаев.

\paragraph{Случай $\sum |C_i| = 4$, $\sum |D_i| = 1$.}~

В первую очередь разберём частный случай, в котором $C_i$ состоят только из литералов 3-переменных.

\begin{brule}
 Если $x$ -- $(3,2)$-переменная в обозначениях выше, $\sum |C_i| = 4$, $\sum |D_i| = 1$, и при этом в объединении $C_i$ содержатся только литералы двух 3-переменных (каждой -- по два), расщепиться по любой из этих 3-переменных.

 Это даёт хотя бы $(9,9)$-расщепление.
 \label{brule:n5:32-rest:c4:3v}
\end{brule}

\begin{proof}
 В силу правила расщепления \ref{brule:n5:32-rest:same-c} и правила упрощения \ref{rrule:common:xu19rr9} остаётся единственный случай взаимного расположения литералов 3-переменных в $C_i$, обозначенный ниже:

 $$
  (x \vee y \vee z) \wedge (x \vee \ovl{y}) \wedge (x \vee \ovl{z}) \wedge \ovl{x} \wedge F'
 $$

 Докажем, что в обоих случаях $y = 1$ и $y = 0$ элиминируются все три переменных $x$, $y$ и $z$.

 В случае $y = 1$ остаются дизъюнкты $x$ и $\ovl{x}$, элиминирующиеся по правилу упрощения \ref{rrule:common:almost-common}, после чего обе переменных $x$ и $z$ являются 2-переменными, также сразу элиминирующимися правилами упрощения \ref{rrule:common:resolution} или \ref{rrule:common:i0}.

 В случае $y = 0$ остаются дизъюнкты $x \vee z$ и $x \vee \ovl{z}$, заменяющиеся по правилу упрощения \ref{rrule:common:almost-common} на дизъюнкт $x$. После этого в формуле остаются дизъюнкты $x$ и $\ovl{x}$, исчезающие по тому же правилу, а $z$ остаётся 1-переменной, элимнинирующейся правилом \ref{rrule:common:i0}.
\end{proof}

Теперь воспользуемся таблицей \ref{table:n5:varcases} для вывода нового правила расщепления.

\begin{brule}
 Если $x$ -- $(3,2)$-переменная в обозначениях выше, $\sum |C_i| = 4$, $\sum |D_i| = 1$, и при этом в объединении $C_i$ или в $D$ содержится переменная, не являющаяся 5-переменной, входящей в это объединение однажды, расщепиться по $x$.

 Это даёт хотя бы $(6,10)$-расщепление.
{\color{white} Вот это грязь, конечно.}
 \label{brule:n5:32-rest:c4:not-5v}
\end{brule}


\begin{proof}
 Лемма \ref{lemma:measure:branch-on} гарантирует в этом случае $(6,9)$-расщепление. При этом, если в объединении $C_i$ или в $D$ есть переменная, чей вес уменьшается больше, чем количество литералов-соседей $x$, с одной из сторон вектор расщепления увеличится и получится либо $(7,9)$, либо $(6,10)$.

 Поскольку $x$ -- $(3,2)$-переменная, ни в одном из этих объединений не может быть переменной, встречающейся четырежды, таким образом, случай 4-переменной, встречающейся четырежды, отпадает.

 Допустим, в объединении $C_i$ есть два литерала одной и той же 3-переменной $y$. Если третье вхождение переменной $y$ находится в $D$, применимо либо правило расщепления \ref{brule:n5:32-rest:same}, либо правило расщеплления \ref{brule:n5:32-rest:compl}. Иначе, поскольку эти вхождения разнознаковые и правило упрощения \ref{rrule:common:unit-clauses} неприменимо, третье вхождение должно быть не в дизъюнкте длины 1, и, следовательно, иметь соседа -- литерал $l$, причём $l$ -- не литерал переменной $x$. Поскольку правило \ref{brule:n5:32-rest:c4:3v} неприменимо, все остальные переменные, входящие в $C_i$ -- 5-переменные, входящие туда однажды. Тогда в случае $x = 1$ этой переменной $y$ будет по правилу \ref{rrule:common:i0} назначено значение и этот литерал элиминируется. Если переменная литерала $l$ входила в $C_i$, то она была 5-переменной, и её вес при удалении ещё одного литерала уменьшится. Если же эта переменная не входила в $C_i$, то от этого её вес тоже уменьшится. Таким образом, мы получаем уменьшение веса ещё хотя бы на 1, и итоговый вектор не хуже, чем $(6,10)$.

 Из векторов $(7,9)$ и $(6,10)$ худшим является второй, таким образом, он и является оценкой на худший случай.
\end{proof}

Наконец, продемонстрируем применение леммы \ref{lemma:n5:3-cases}.

\begin{brule}
 Пусть $x$ -- $(3,2)$-переменная в обозначениях выше, $\sum |C_i| = 4$, $\sum |D_i| = 1$, и при этом предыдущие правила к ней неприменимы. Пусть, не умаляя общности, $|C_1| = 2$. Тогда расщепиться на три случая:

 \begin{enumerate}
  \item $x = 1$
  \item $x = 0$, $C_2 = 1$
  \item $x = 0$, $C_2 = 0$, $C_3 = 1$, $D_1 = 0$.
 \end{enumerate}

 Это даёт хотя бы $(9,11,20)$-расщепление.
 \label{brule:n5:32-rest:c4:3-cases}
\end{brule}

\begin{proof}
 Корректность правила доказана в лемме \ref{lemma:n5:3-cases}.

 Отметим отдельно, что в силу правил \ref{brule:n5:32-rest:same}, \ref{brule:n5:32-rest:compl} и \ref{brule:n5:32-rest:same-c} в $C_2$, $C_3$ и $D_1$ все переменные различны, а в силу правила \ref{brule:n5:32-rest:c4:not-5v}, все они -- 5-переменные.

 В первом случае по лемме \ref{lemma:measure:branch-on} мера уменьшается хотя бы на 9.

 Во втором случае исчезают 5-переменные $x$ и $C_2$, а также литерал другой переменной в $D_1$. Таким образом, мера уменьшается хотя бы на 11.

 В третьем случае исчезают четыре различных 5-переменных $x$, $C_2$, $C_3$ и $D_1$, таким образом, мера уменьшается хотя бы на 20.
\end{proof}

Это правило завершает разбор случая $\sum |C_i| = 4$.

\paragraph{Случай $\sum |C_i| = 3$, $1 \leq \sum |D_i| \leq 2$.} ~

Разбор имеет похожую структуру, но без первого правила.

Прежде всего заметим, что в $C_i$ в силу правила расщепления \ref{brule:n5:32-rest:same-c} и правила упрощения \ref{rrule:common:almost-common} все переменные различные, а в силу правил расщепления \ref{brule:n5:32-rest:same} и \ref{brule:n5:32-rest:compl} эти переменные не встречаются в $D$.

\begin{brule}
 Пусть $x$ -- $(3,2)$-переменная в обозначениях выше, $\sum |C_i| = 3$ и при этом выполняется одно из двух условий.

 \begin{enumerate}
  \item $\sum |D_i| = 2$, и в объединении $C_i$ и $D_i$ есть переменная, не являющаяся 5-переменной, встречающейся однажды.
  \item $\sum |D_i| = 1$, и в объединении $C_i$ и $D_i$ более одной $4^-$-переменной.
 \end{enumerate}

 Тогда расщепиться по $x$.

 Это даёт хотя бы $(6,10)$-расщепление.
 \label{brule:n5:32-rest:c3:not-5v}
\end{brule}

\begin{proof}
 По случаям.
 \begin{enumerate}
  \item Лемма \ref{lemma:measure:branch-on} гарантирует хотя бы $(7,8)$-расщепление. В условии же гарантируется существование либо 3-переменной, встречающейся дважды, либо переменной, улучшающей указанный вектор. Во втором случае будет хотя бы вектор $(8,8)$ или $(7,9)$. Первый случай невозможен, так как множества переменных в объединениях $C_i$ и $D_i$ не пересекаются, два одинаковых литерала 3-переменной с одной стороны невозможны по правилу упрощения \ref{rrule:common:xu19rr9}, а два разных литерала невозможны по правилу упрощения \ref{rrule:common:almost-common}.

  \item Отметим, что все переменные в объединении в таком случае различны. Тогда лемма \ref{lemma:measure:branch-on} гарантирует хотя бы $(6,8)$-расщепление. В таблице \ref{table:n5:varcases} встречающиеся однажды $4^-$-переменные дают дополнительное уменьшение меры на 1. Две таких переменных дадут уменьшение на 2, то есть в итоге получится $(8,8)$-, $(7,9)$- или $(6,10)$-расщепление.
 \end{enumerate}

 Худшим из представленных векторов является вектор $(6,10)$, таким образом, он и является оценкой на худший случай.
\end{proof}

В оставшемся случае применим лемму \ref{lemma:n5:3-cases}.

\begin{brule}
 Пусть $x$ -- $(3,2)$-переменная в обозначениях выше, $\sum |C_i| = 3$, и при этом предыдущие правила к ней неприменимы. Пусть, не умаляя общности, $C_1$ содержит в себе литерал 5-переменной. Тогда расщепиться на три случая:

 \begin{enumerate}
  \item $x = 1$
  \item $x = 0$, $C_1 = 1$
  \item $x = 0$, $C_1 = 0$, $C_2 = C_3 = 1$, $D_1 = D_2 = 0$.
 \end{enumerate}

 Это даёт хотя бы $(8,11,24)$-расщепление.
 \label{brule:n5:32-rest:c3:3-cases}
\end{brule}

\begin{proof}
 Корректность правила доказана в лемме \ref{lemma:n5:3-cases}.
 
 Далее рассмотрим оставшиеся после правила \ref{brule:n5:32-rest:c3:not-5v} случаи.

 \begin{itemize}
  \item $\sum |D_i| = 2$. Тогда все переменные в объединении $C_i$ и $D$ -- различные 5-переменные. В первом случае по лемме \ref{lemma:measure:branch-on} мера уменьшается хотя бы на 8. Во втором случае означиваются 5-переменные $x$ и $C_1$, а также исчезают два литерала других 5-переменных в $D_1$ и $D_2$, таким образом, мера уменьшается хотя бы на 12. В третьем случае означиваются шесть 5-переменных: $x$, три из $C_i$ и две из $D_i$, и мера уменьшается хотя бы на 30. Итоговый вектор: $(8,12,30)$.

  \item $\sum |D_i| = 1$, все переменные в объединении $C_i$ и $D$ -- $4^+$-переменные. После правила расщепления \ref{brule:n5:32-rest:c3:not-5v} 4-переменной из них может быть лишь одна. В первом случае мера по лемме \ref{lemma:measure:branch-on} уменьшается хотя бы на 8. Во втором случае означиваются две 5-переменные, и исчезат один литерал другой переменной, таким образом, мера уменьшается хотя бы на 11. В третьем случае означиваются четыре 5-переменных и $4^+$-переменная, и мера уменьшается хотя бы на 24. Итоговый вектор: $(8,11,24)$.

  \item $\sum |D_i| = 1$, 3-переменная находится в одном из $C_i$. Тогда она элиминируется в первом случае полностью и мера в первом случае уменьшается хотя бы на 9. Во втором случае означиваются две 5-переменные, и исчезает один литерал другой переменной, таким образом, мера уменьшается дотя бы на 11. В третьем случае означиваются четыре 5-переменных и 3-переменная, и мера уменьшается хотя бы на 22. Итоговый вектор: $(9,11,22)$.

  \item $\sum |D_i| = 1$, 3-переменная находится в $D_1$. В первом случае мера по лемме \ref{lemma:measure:branch-on} уменьшается хотя бы на 8. Во втором случае означиваются две 5-переменные, а также полностью элиминируется указанная 3-переменная, таким образом, мера уменьшается хотя бы на 12. В третьем случае означиваются четыре 5-переменных и 3-переменная, и мера уменьшается хотя бы на 22. Итоговый вектор: $(8,12,22)$.
 \end{itemize}

 Худшим из представленных векторов является вектор $(8,11,24)$, таким образом, он и является оценкой на худший случай.
\end{proof}

Это правило заканчивает разбор и случая $\sum |C_i| = 3$, и разбор всех 5-переменных.

В качестве резюме к разделу \ref{subsec:n5:32-rest}, здесь фактически доказана следующая теорема, которая помогает взглянуть на приведённые правила с практического угла.

\begin{theorem}
 Пусть $x$ -- $(3,2)$-переменная, входящая не более чем в один дизъюнкт длины 1 лишь отрицательно. Тогда одно из следующих расщеплений даёт вектор хотя бы $(6,10)$:

 \begin{enumerate}
  \item Расщепление по $x$.
  \item Расщепление по одному из соседей $x$.
  \item Расщепление по лемме \ref{lemma:n5:3-cases} (если она применима).
  \item Расщепление по лемме \ref{lemma:n5:2-cases} (если она применима).
 \end{enumerate}

 \label{theorem:n5:32-rest}
\end{theorem}

Фактически, в разборе случаев в этом разделе описано доказательство этой теоремы.

\subsection{Выводы}
\label{subsec:n5:summary}

\begin{itemize}
 \item Продемонстрированы свойства меры $d$ при разборе $5^-$-переменных.
 \item Представлены правила расщепления для $(3,2)$-переменных.
 \item На основании этих свойств и правил представлены правила, элиминирующие все 5-переменные в формуле, с худшим вектором расщепления $(6,10)$.
\end{itemize}

% vim: spell spelllang=ru_yo,en_gb
