\newgeometry{top=2cm,bottom=2cm,left=3cm,right=1.5cm,nohead,includeheadfoot}

\section*{Заключение}
\label{sec:conclusion}
\addcontentsline{toc}{section}{\nameref{sec:conclusion}}

\renewcommand{\textflush}{flushepinormal}
\begin{otherlanguage}{latin}
 \epigraph{Itaque earum rerum hic tenetur a sapiente delectus, ut aut reiciendis voluptatibus maiores alias consequatur aut perferendis doloribus asperiores repellat.}{M. Tullius Cicero}
\end{otherlanguage}

Основным результатом работы является следующая теорема. По сравнению с предыдущим лучшим результатом в области \cite{bansal99} она даёт улучшение в логарифмической шкале на 11.7\%.

\setcounter{section}{5}
\setcounter{lemma}{0}
\begin{theorem}
 Существует алгоритм для задачи MAXSAT, работающий за $O^*(1.0927^L)$.
\end{theorem}

\begin{proof}
 Как показано в разделе \ref{subsec:measure:motivation}, любой алгоритм за $O^*(\alpha^d)$ работает и за $O^*(\alpha^L)$.

 Алгоритм за $O^*(1.0927^d)$ представлен в главах \ref{sec:measure} -- \ref{sec:n4}. Он последовательно выполняет правила упрощения и расщепления до тех пор, пока формула не станет экземпляром задачи $(n,3)$-MAXSAT, после чего запускается алгоритм, представленный в работе \cite{belova18} для этой задачи.

 Полученные оценки на вектора расщепления представлены для удобства в таблице \ref{table:conclusion:vectors}. Последняя строчка относится к работе \cite{belova18}. Как видно, худшим числом расщепления среди представленных правил является число $1.0927$, таким образом, представленный алгоритм работает за $O^*(1.0927^L)$.
\end{proof}

\begin{table}[ht]
 \centering
 \caption{Полученные оценки на вектора расщепления}
 \begin{tabular}{|l|c|c|}
  \hline
  \multicolumn{1}{|c|}{\textbf{Случай}} & \textbf{Вектор} & \textbf{Число} \\
  \hline
  Правило расщепления \ref{brule:measure:sixplus}
          & $(6,10)$
          & 1.0927 \\
  Правило расщепления \ref{brule:n5:41}
          & $(7,9)$
          & 1.0911 \\
  Правило расщепления \ref{brule:n5:32-2uc}
          & $(5,12)$
          & 1.0908 \\
  Правило расщепления \ref{brule:n5:32-+uc}
          & $(7,9)$
          & 1.0911 \\
  Правило расщепления \ref{brule:n5:32-rest:length}
          & $(6,10)$
          & 1.0927 \\
  Правило расщепления \ref{brule:n5:32-rest:same}
          & $(8,8)$
          & 1.0906 \\
  Правило расщепления \ref{brule:n5:32-rest:compl}
          & $(7,9)$
          & 1.0911 \\
  Правило расщепления \ref{brule:n5:32-rest:same-c}
          & $(8,9)$
          & 1.0851 \\
  Правило расщепления \ref{brule:n5:32-rest:c4:3v}
          & $(9,9)$
          & 1.0801 \\
  Правило расщепления \ref{brule:n5:32-rest:c4:not-5v}
          & $(6,10)$
          & 1.0927 \\
  Правило расщепления \ref{brule:n5:32-rest:c4:3-cases}
          & $(9,11,20)$
          & 1.0925 \\
  Правило расщепления \ref{brule:n5:32-rest:c3:not-5v}
          & $(6,10)$
          & 1.0927 \\
  Правило расщепления \ref{brule:n5:32-rest:c3:3-cases}
          & $(8,11,24)$
          & 1.0916 \\
  Правило расщепления \ref{brule:n4:22:nouc}
          & $(8,8)$
          & 1.0906 \\
  Правило расщепления \ref{brule:n4:22:uc}
          & $(6,10)$
          & 1.0927 \\
  Правило расщепления \ref{brule:n4:31:ns}
          & $(6,10)$
          & 1.0927 \\
  Правило расщепления \ref{brule:n4:31:2c}
          & $(6,10)$
          & 1.0927 \\
  Правило расщепления \ref{brule:n4:31:4v-2}
          & $(4,14)$
          & 1.0912 \\
  Правило расщепления \ref{brule:n4:31:final}
          & $(4,14)$
          & 1.0912 \\
  \hline
  \multicolumn{1}{|c|}{$(n,3)$-MAXSAT}
          & --
          & 1.0912 \\
  \hline
 \end{tabular}
 \label{table:conclusion:vectors}
\end{table}

Также в работе введена уменьшенная длина $d = L - n_3$ -- мера сложности задачи, более сбаллансированная по сравнению с длиной. Для неё продемонстрированы следующие свойства:
\begin{itemize}
 \item Показано, что при расщеплении по переменным большой кратности она ведёт себя как длина. Конкретнее -- в лемме \ref{lemma:measure:branch-on} для неё доказана гарантия вектора расщепления, такая же, как базовая гарантия для длины.
 \item Для 5-переменных продемонстрировано, что эта оценка почти всегда улучшается за счёт свойств меры на переменных малой кратности, а в случае, если такое улучшение недостаточно, показано, что такая 5-переменная имеет много соседей большой кратности, что также позволяет строить эффективные правила расщепления.
 \item Для 4-переменных в леммах \ref{lemma:n4:one-side} и \ref{lemma:n4:branch-on} показано, что уменьшенная длина ведёт себя похоже на параметризацию количеством переменных задачи $(n,3)$-MAXSAT. Это позволяет строить эффективные правила расщепления и упрощения.
 \item Благодаря указанным выше свойствам, мера уравнивает времена работы алгоритма в худшем случае для 4-, 5- и 6-переменных и почти доводит их до времени работы алгоритма $(n,3)$-MAXSAT.
\end{itemize}

Дальнейшая работа представляет интерес в следующих направлениях:

\begin{itemize}
 \item В данной работе была выбрана относительно простая мера $d = L - n_3$, все значения которой целочислены. С одной стороны, это сильно упрощает работу с мерой, но с другой -- сильно ограничивает возможности по применению такой меры. Например, в отсутствие продвижения алгоритмов для задачи $(n,3)$-MAXSAT алгоритм относительно такой меры ограничен снизу асимптотикой $O^*(1.0912^L)$ -- временем работы для $(n,3)$-MAXSAT, что очень близко к полученной в этой работе верхней оценке. При этом для выполнения во всяком случае свойства меры на переменных большой кратности (леммы \ref{lemma:measure:branch-on}) достаточно, чтобы веса переменных с кратностью, отличающейся на единицу, отличались между собой хотя бы на единицу, и вес 3-переменных был хотя бы 2. Таким образом, интерес представляют меры вида $d(\alpha) = L - \alpha n_3$ где $\alpha \in [0; 1]$. При $\alpha = 0$ это выражение вырождается в длину, а при $\alpha = 1$ -- в уменьшенную меру. По мнению автора, разработка такой меры может дать лучший результат по сравнению с мерой $d$.

 \item При выполнении работы стало понятно, что большая часть разбора случаев, выполненного в итоге вручную, скорее всего, может быть выполнена так или иначе автоматически: в алгоритме используется ограниченный набор схем для правил расщепления и упрощения. Представляет интерес возможность автоматизированного разбора случаев, что может позволить спускаться в разборе гораздо ниже, чем это возможно при ручной работе. В комбинации с предыдущим направлением это может позволить выбрать оптимальный параметр $\alpha$, автоматически разобрав все случаи. При этом нужно понимать, что структура дерева разбора случаев будет зависеть от $\alpha$. Однако, используя численные методы, можно приблизить оптимальное значение и использовать его, например, для ручного доказательства оставшейся части алгоритма.
\end{itemize}

% vim: spell spelllang=ru_yo,en_gb ft=tex tabstop=1 shiftwidth=1
