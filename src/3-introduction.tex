\newgeometry{top=2cm,bottom=2cm,left=3cm,right=1.5cm,nohead,includeheadfoot}

\section*{Введение}
\label{sec:intro}
\addcontentsline{toc}{section}{\nameref{sec:intro}}

\subsection*{Актуальность задачи}

\firstpar{}Задача о максимальной разрешимости, или, сокращённо, MAXSAT, как оптимизационная версия задачи о разрешимости (сокращённо SAT), возможно, одной из самых известных NP-полных задач, имеет широкий круг применений, от анализа данных \cite{berg2015applications} до медицины \cite{lin2012application}.
При этом не только задача MAXSAT, но многие её частные случаи, такие, как $(n,3)$-MAXSAT, являются NP-трудными \cite{raman1998simplified}.

Гипотеза об экспоненциальном времени говорит, что задача 3SAT, то есть задача булевой разрешимости с дополнительным ограничением, что длина каждого дизюнкта не более трёх, не может быть решена быстрее, чем за экспоненциальное время от количества переменных и тем более от длины входа. Как следствие, задача о максимальной разрешимости также не может быть решена за субэкспоненциальное время от длины входа, так как наличие такого алгоритма автоматически влекло бы за собой существование алгоритма дла 3-SAT. Поэтому основное направление исследований в этой области -- уменьшение основания экспоненты.

В последние годы активно продвинулось изучение других параметризаций той же задачи: параметризация количеством выполненых дизъюнктов \cite{chen15} и общим количеством дизъюнктов \cite{xu19}. Для формул с большой средней длиной дизъюнкта эти алгоритмы дают хорошее время работы. Однако если в формуле большинство дизъюнктов имеют длину 1 или 2, алгоритм относительно длины применять эффективнее. При этом стоит отметить, что задача MAX-2-SAT, где все дизъюнкты имеют длину 1 или 2, уже является NP-трудной, хотя для неё существуют специальные алгоритмы, позволяющие решать её быстрее, чем в общем случае \cite{golovnev2014new}. Тем не менее, если ограничения на максимальную длину дизъюнкта нет, но средняя длина небольшая, задача эффективно решается именно алгоритмом относительно длины формулы.

\subsection*{Условие задачи}

\firstpar{}Задача MAXSAT формулируется следующим образом:

\begin{center}
 \begin{tabular}{|lp{.8\textwidth}|}
  \hline
  \multicolumn{2}{|l|}{MAXSAT} \\
  \textbf{Вход:} & Булева формула $F$ в конъюнктивной нормальной форме (КНФ) и число $k$ \\
  \textbf{Ответ:} & Означивание переменных, выполняющее хотя бы $k$ дизъюнктов. \\
  \hline
 \end{tabular}
\end{center}

Длина формулы обозначается за $L$.

Как упоминалось выше, цель работы -- создать алгоритм за $O^*(\alpha^L)$ при минимальном $\alpha$.
Алгоритм будет иметь следующую структуру:

\begin{enumerate}
 \item Свести экземпляр задачи к экземпляру задачи $(n,5)$-MAXSAT.
 
 \item Свести экземпляр задачи к экземпляру задачи $(n,3)$-MAXSAT.

 \item Запустить на полученном экземпляре лучший известный алгоритм для $(n,3)$-MAXSAT \cite{belova18}.
\end{enumerate}

Данная структура формирует представление о параметризациях, используемых на разных этапах алгоритма. На первом этапе мы решаем задачу относительно длины формулы. На втором этапе мы используем новый специально введённый нами параметр, уменьшающий стоимость 3-переменных, позволяющий использовать замечательное время работы алгоритма для $(n,3)$-MAXSAT. Наконец, на третьем этапе, мы используем естественную для $(n,3)$-MAXSAT параметризацию количеством переменных.

\subsection*{Ограничения работы}

\firstpar{}Алгоритм для задачи булевой разрешимости (SAT), работающий за $O^*(1.074^L)$, был представлен Гиршем в 2000 году \cite{hirsch2000new}. Несмотря на то, что целью работы является подойти ближе к этой границе, едва ли удастся её преодолеть, так как существование лучшего алгоритма для MAXSAT повлекло бы существование лучшего алгоритма для более простой задачи SAT. В частности, все худшие случаи представленного алгоритма в задаче булевой разрешимости разбирались бы тривиально.

В данной работе рассматривается лишь алгоритм относительно длины формулы, но не другие варианты параметризации задачи MAXSAT.

\subsection*{Определения ключевых терминов}

Булевы переменные в работе обозначаются буквами $x$, $y$, $z$, $w$.

Если $x$ — булева переменная, то выражения $x$ и $\ovl{x}$ называются литералами. Во избежание неоднозначности литералы в работе обозначаются буквами $l$, $k$, $m$.

Дизъюнкт — это дизъюнкция литералов, то есть выражение вида $x_1 \vee \ovl{x_2} \vee x_3 \vee \ldots$. По умолчанию считается, что повторяющихся литералов в дизъюнкте нет, иначе их можно было бы сократить по правилу $l \vee l = l$. В работе дизъюнкты обозначаются буквами $C$, $D$, $E$.

Формула находится в конънктивной нормальной форме (КНФ), если она является конъюнкцией дизъюнктов (то есть имеет вид $(x_1 \vee \ovl{x_2} \vee x_3) \wedge \ovl x_1 \wedge \ldots$).

Задача булевой разрешимости состоит в том, чтобы определить, существует ли означивание переменных, выполняющее формулу в КНФ. У неё есть вариант $k$-SAT с дополнительным ограничением на длину каждого дизъюнкта: она не больше $k$. В то время как задача 3SAT уже является NP-трудной, для задачи 2SAT известно полиномиальное решение.

Задача о максимальной разрешимости, как сформулировано выше, является оптимизационной версией этой задачи, и требует выполнения не всех, но хотя бы заданного числа дизъюнктов. У неё есть аналогичные варианты MAX-$k$-SAT. Кроме того, выделяют версии этой задачи $(n,k)$-MAXSAT, в которых каждая переменная входит в формулу не более $k$ раз. При этом задачи MAX-2-SAT, $(n,3)$-MAXSAT и даже $(n,3)$-MAX-2-SAT уже являются NP-трудными \cite{raman1998simplified}. Для задачи $(n,2)$-MAXSAT существует полиномиальное решение.

Переменная называется $k$-переменной, если она входит в формулу ровно $k$ раз.
Если про переменную известно, что она входит в формулу хотя бы $k$ раз, она называется $k^+$-переменной. Аналогично, если известно, что переменная входит в формулу не более $k$ раз, она называется $k^-$-переменной. Число $k$-переменных в формуле обозначается за $n_k$.

Если переменная $x$ входит $k$ раз положительно (то есть как литерал $x$) и $l$ раз отрицательно (как литерал $\ovl{x}$), она называется $(k,l)$-переменной. Такое же обозначение вводится и для литералов. Поскольку замена переменной $x$ на $\ovl{x}$ во всей формуле не влияет на ответ на задачу, если не указано иного, при обозначении переменных всегда считается $k \geq l$.

Алгоритм состоит из правил сокращения и правил расшепления.

Правило сокращения — полиномиальный алгоритм, преобразующий экземпляр задачи в эквивалентный ему и при этом не увеличивающий длину формулы. Такие правила применяются к формуле постоянно, пока это возможно. В правилах расщепления считается, что ни одно правило сокращения к формуле неприменимо.

Правило расщепления — полиномиальный алгоритм, преобразующий экземпляр задачи в несколько других вариантов и при этом с необходимостью уменьшающий длину формулы в каждом из них. От каждого из них алгоритм запускается рекурсивно, если хотя бы в одном из них удаётся получить положительный ответ, ответ на задачу является положительным.

Если во вариантах, произведённых правилом расщепления, длина уменьшается на $a_1, \dots, a_k$, то время работы алгоритма оценивается как рекуррентное соотношение

$$
 T(L) = T(L - a_1) + \dots + T(L - a_k)
$$

Решением такого соотношения является $T(n) = O^*(\alpha^n)$, где $\alpha$ -- единственный больший единицы корень уравнения

$$
 1 = \alpha^{-a_1} + \dots + \alpha^{-a_k}
$$

Вектор $(a_1, \dots, a_k)$ называется вектором расщепления, а число $\alpha$ -- числом расщепления. Асимптотика всего алгоритма оценивается как $O^*(\alpha^L)$, где $\alpha$ -- максимальное по всем правилам расщепления число расщепления.

Расщепление по переменной $x$ -- правило расщепления, разделяющееся на случаи $x = 0$ и $x = 1$. Разумеется, такое правило исчерпывает все варианты и по ответам в каждом из вариантов можно восстановить ответ на исходную формулу.

Подформула -- это подмножество дизъюнктов исходной формулы. Подформула называется замкнутой, если все переменные, литералы которых содержатся в подформуле, не имеют литералов вне этой подформулы.

\subsection*{Полученные результаты}

\firstpar{}Верхняя оценка на задачи максимальной разрешимости улучшена в данной работе с $O^*(1.106^L)$ до $O^*(1.093^L)$. В логарифмической шкале это даёт улучшение с $O^*(2^{0.145L})$ до $O^*(2^{0.128L})$, то есть на 11.7\%.

Кроме того, в работе предложена новая параметризация для задачи MAXSAT с уменьшенной стоимостью 3-переменных. Продолжение исследований в этом направлении может помочь сдвинуть эту границу ещё дальше.

% vim: spell spelllang=ru_yo,en_gb
