\newgeometry{top=2cm,bottom=2cm,left=3cm,right=1.5cm,nohead,includeheadfoot}

\section{Обзор литературы}
\label{sec:literature-review}

\subsection{История развития области}
\label{subsec:literature-review:history}

\firstpar{}Задача булевой разрешимости была исторически первой задачей, для которой была доказана NP-полнота (этот известный факт называется теоремой Кука-Левина). Задача о максимальной разрешимости, как оптимизационная версия этой задачи, автоматически является NP-полной и изучается с тех времён.

История развития точных алгоритмов для MAXSAT представлена в таблице \ref{table:maxsat-length-research}.

\begin{table}[ht]
 \caption{История развития алгоритмов для задачи MAXSAT}
 \centering
 \begin{tabular}{|c|c|c|c|}
  \hline
  \textbf{Работа} & \textbf{Год} & \textbf{Результат} & \textbf{Δ} \\
  \hline
  Нидермайер и Россманит \cite{niedermeier1999new} & \citeyear{niedermeier1999new} & $O^*(1.1279^L) = O^*(2^{0.1737L})$ & -- \\
  Банзал и Раман \cite{bansal99} & \citeyear{bansal99} & $O^*(1.1057^L) = O^*(2^{0.1450L})$ & 16.5\% \\
  \hline
 \end{tabular}
 \label{table:maxsat-length-research}
\end{table}

Видно, что со времён работы \cite{bansal99} значимого улучшения не произошло. Причиной тому является недостаточное развитие смежного алгоритма для $(n,3)$-MAXSAT.

Как будет продемонстрированно в данном обзоре чуть ниже, простые правила сокращения позволяют убрать из формулы 1- и 2-перменные, таким образом. можно считать, что все переменные в $(n,3)$-MAXSAT-формуле являются 3-переменными, а длина формулы, таким образом, равняется утроенному количеству переменных. Одним из направлений исследований алгоритмов для задачи $(n,3)$-MAXSAT является исследование алгоритмов, экспоненциальных относительно количества переменных в формуле. В силу равненства $L = 3n$ таковой алгоритм автоматически является и алгоритмом относительно длины. История этих алгоритмов приведена в таблице \label{table:n3-maxsat-research}.

\begin{table}[ht]
 \caption{История развития алгоритмов для задачи $(n,3)$-MAXSAT}
 \centering
 \begin{tabular}{|c|c|c|}
  \hline
  \textbf{Работа} & \textbf{Год} & \textbf{Результат} \\
  \hline
  Раман, Равикумар и Рао \cite{raman1998simplified} & \citeyear{raman1998simplified} & $O^*(1.732^n)$ \\
  Банзал и Раман \cite{bansal99} & \citeyear{bansal99} & $O^*(1.3248^n)$ \\
  Куликов и Куцков \cite{kulikov2009new} & \citeyear{kulikov2009new} & $O^*(1.2721^n)$ \\
  Близнец \cite{bliznets2013new} & \citeyear{bliznets2013new} & $O^*(1.2600^n)$ \\
  Чэнь, Сюй и Ван \cite{chen15} & \citeyear{chen15} & $O^*(1.237^n)$ \\
  Ли, Cюй, Ван и Ян \cite{li2017improved} & \citeyear{li2017improved} & $O^*(1.194^n)$ \\
  Белова и Близнец \cite{belova18} & \citeyear{belova18} & $O^*(1.191^n)$ \\
  \hline
 \end{tabular}
 \label{table:n3-maxsat-research}
\end{table}

Отметим, что для достижения заявленной нами асимптотики с помощью введённого нами параметра, необходимо существование алгоритма для $(n,3)$-MAXSAT, работающего не хуже, чем за время $O^*(1.194^n)$.

В свою очередь, активное развитие алгоритмов для $(n,3)$-MAXSAT стало возможным благодаря представленной Близнецом и Головнёвым \cite{bliznets12} идеи сведения задачи к задаче о покрытии множества. Эта идея первоначально была применена к задаче о максимальной разрешимости, параметризованной ответом. История работ, связанных с этой формулировкой задачи, приведены в таблице \ref{table:maxsat-answer-research}.

\begin{table}[ht]
 \caption{Развитие алгоритмов для MAXSAT относительно ответа}
 \centering
 \begin{tabular}{|c|c|c|c|}
  \hline
  \textbf{Работа} & \textbf{Год} & \textbf{Результат} & \textbf{Δ} \\
  \hline
  Махаджан и Раман \cite{mahajan1999parameterizing} & \citeyear{mahajan1999parameterizing} & $O^*(1.618^k) = O^*(2^{0.695k})$ & -- \\
  Нидермайер и Россманит \cite{niedermeier1999new} & \citeyear{niedermeier1999new} & $O^*(1.400^k) = O^*(2^{0.486k})$ & 30\% \\
  Банзал и Раман \cite{bansal99} & \citeyear{bansal99} & $O^*(1.381^k) = O^*(2^{0.466k})$ & 4\% \\
  Чэнь и Кандж \cite{chen2004improved} & \citeyear{chen2004improved} & $O^*(1.370^k) = O^*(2^{0.455k})$ & 2.5\% \\
  Близнец и Головнёв \cite{bliznets12} & \citeyear{bliznets12} & $O^*(1.358^k) = O^*(2^{0.442k})$ & 2.8\% \\
  Чэнь и Сюй \cite{chen15} & \citeyear{chen15} & $O^*(1.325^k) = O^*(2^{0.406^k})$ & 8\% \\
  \hline
 \end{tabular}
 \label{table:maxsat-answer-research}
\end{table}

Отдельно хочется отметить развитие алгоритмов для задачи о максимальной разрешимости, параметризованной общим количеством дизъюнктов. Лучший результат в этой области $O^*(1.2989^m)$ был получен Сюй и др. \cite{xu19} в \citeyear{xu19} году. В то время как эта работа также использует идею сведения к задаче о покрытии множества, там также введено большое количество новых правил сокращения. В то время как большая их часть увеличивает длину формулы (и, следовательно, не может быть применена к рассматриваемой задаче), одно из них оказывается весьма полезным. Подробнее об этом рассказано в данном обзоре ниже.

\subsection{Правила сокращения}
\label{subsec:literature-review:rrules}

\firstpar{}Правила сокращения -- одно из самых мощных средств построения алгоритмов благодаря тому, что многие из них можно переиспользовать для разных версий одной и той же задачи. В данном разделе представлены правила сокращения из литературы, используемые в представленном в работе алгоритме.

Отметим, что, несмотря на то, что формальная постановка задачи о максимальной разрешимости подразумевает, что во входных данных содержится требуемое количество выполненных дизъюнктов $k$, для краткости это количество будет опускаться. Это возможно благодаря тому, что представленный алгоритм сразу строит означивание, выполняющее наибольшее возможное количество дизъюнктов, игнорируя число $k$ до момента ответа.

\begin{rrule}
 Если для переменной $x$ оба литерала $x$ и $\ovl{x}$ содержатся в одном дизъюнкте $x \vee \ovl{x} \vee C$, то можно удалить этот дизъюнкт.
 \label{rrule:common:complementary}
\end{rrule}

Это правило корректно, поскольку выражение $x \vee \ovl{x}$ верно при любом означивании переменных, и, следовательно, дизънкт $x \vee \ovl{x} \vee C$ выполняется всегда. Удаление дизъюнкта не увеличивает длины формулы.

Правило является очевидным и относится больше к вопросу формального определения формулы, нежели к построению алгоритма. Так, в работе \cite{bansal99} это правило опускается, так как определение понятия дизъюнкта, используемое там, исключает возможность ситуации, в которой такое правило применимо.

\begin{rrule}
 Пусть в формуле содержится литерал $l$ такой, что литерал $\ovl{l}$ в формуле отсутствует. Тогда можно означить $l = 1$.
 \label{rrule:common:i0}
\end{rrule}

Правило снова является очевидным: при переходе от означивания $l = 0$ к означиванию $l = 1$ ни один дизъюнкт не перестаёт быть выполненным, а хотя бы один новый дизъюнкт, напротив, становится выполненным. В силу очевидности не вполне корректно приводить ссылки на конкретную работу, где оно введено.

\begin{rrule}[Almost common clauses, \cite{bansal99}]
 Пусть для некоторой переменной $x$ и (вомзможно, пустого) дизъюнкта $C$ оба дизъюнкта $x \vee C$ и $\ovl{x} \vee C$ входят в формулу. Тогда оба этих дизъюнкта можно заменить на один дизъюнкт $C$.
 \label{rrule:common:almost-common}
\end{rrule}

Это правило встречается в работе \cite{bansal99}. Вариация с пустым $C$ была известна и ранее.

\begin{note}
 В частности, после того, как правило \ref{rrule:common:almost-common} неприменимо, для каждой переменной $x$ лишь один из литералов $x$ и $\ovl{x}$ может входить в дизъюнкты длины 1.
\end{note}

\begin{rrule}[Правило резолюций]
 Пусть $x$ -- $(1,1)$-переменная, входящая в дизъюнкты $x \vee C$ и $\ovl{x} \vee D$. Тогда оба этих дизъюнкта можно заменить на один дизъюнкт $C \vee D$.
 \label{rrule:common:resolution}
\end{rrule}

Это правило имеет истоки в пропозициональной логике и было, по видимому, в каком-то виде известно ещё до формулировки задачи булевой выполнимости. В представленном виде оно верно и для задачи о максимальной разрешимости.

\begin{note}
 После того, как правила \ref{rrule:common:i0} и \ref{rrule:common:resolution} неприменимы, в формуле нет $2^-$-переменных, так как все эти переменные элиминируются одним из указанных правил. В частности, задача $(n,2)$-MAXSAT решается за полиномиальное время.
\end{note}

\begin{rrule}[\cite{niedermeier1999new}]
 Пусть $l$ -- $(i,j)$-литерал, входящий в $k$ дизъюнктов длины 1, причём $k \geq j$. Тогда можно назначить $l = 1$.
 \label{rrule:common:unit-clauses}
\end{rrule}

Мощность этого правила заключается в том, что оно ограничивает сверху количество дизъюнктов длины 1, в которое может входить переменная, и, следовательно, ограничивает снизу суммарную длину дизъюнктов, в которые переменная входит. Кроме того, это правило существенно уменьшает разбор случаев.

\begin{rrule}[Правило 9 из \cite{xu19}]
 Пусть $i \geq 2$ и $x$ -- $(i,1)$-переменная, такая, что все дизъюнкты, содержащие $x$, содержат также один и тот же литерал $l$. Тогда можно убрать $l$ из всех этих дизъюнктов и добавить его в дизъюнкт, содержащий $\ovl{x}$. То есть $(x \vee l \vee C_1) \wedge \dots \wedge (x \vee l \vee C_i) \wedge (\ovl{x} \vee D) \wedge F' \rightarrow (x \vee C_1) \wedge \dots \wedge (x \vee C_i) \wedge (\ovl{x} \vee l \vee D) \wedge F'$.
 \label{rrule:common:xu19rr9}
\end{rrule}

Это правило даёт очень мощные ограничения на 3-переменные (каждая из которых после правила \ref{rrule:common:i0} является $(2,1)$-переменной), особенно в сочетании с правилом \ref{rrule:common:almost-common}. Такое сочетание во многих случаях будет ограничивать количество литералов одной и той же 3-переменной в наборы дизъюнктов одним вхождением.

\begin{rrule}
 Пусть в формуле есть замкнутая подформула на не более чем пяти переменных. Тогда эту подформулу можно решить за полиномиальное время независимо от остальной части формулы.
\end{rrule}

Заметим, что замкнутые подформулы можно находить за полиноимальное время, построив граф с вершинами -- переменными из формулы и рёбрами, если концы входят в один дизъюнкт, и найдя в нём компоненты связности. Также заметим, что замкнутые формулы на константном количестве переменных решаются за полиномиальное время выбором произвольной переменной и расщеплением по ней: размер дерева рекурсии получается константным и в каждой вершине выполняется полиномиальное действие.

\subsection{Выводы}
\label{subsec:literature-review:summary}

\begin{itemize}
 \item Работа над задачей остановилась на работе \cite{bansal99} из-за отсутствия алгоритмов для задачи $(n,3)$-MAXSAT с достаточно хорошим временем работы. Такой алгоритм был получен в работах \cite{li2017improved} и \cite{belova18}.
 \item Этому способствовала работа в других параметризациях задачи о максимальной разрешимости, в частности, идея о сведении к задаче о покрытии множества, высказанная в работе \cite{bliznets12}.
 \item Также современные правила сокращения, такие, как правило \ref{rrule:common:xu19rr9}, позволяют сильно уменьшить пространство разбираемых случаев.
\end{itemize}

% vim: spell spelllang=ru_yo,en_gb
