\newgeometry{top=2cm,bottom=2cm,left=3cm,right=1.5cm,nohead,includeheadfoot}

\section{Обзор литературы}
\label{sec:literature-review}

\subsection{История развития области}
\label{subsec:literature-review:history}

\firstpar{}Задача булевой разрешимости была исторически первой задачей, для которой была доказана NP-полнота (этот известный факт называется теоремой Кука-Левина). Задача о максимальной разрешимости, как оптимизационная версия этой задачи, автоматически является NP-полной и изучается с тех времён.

История развития точных алгоритмов для MAXSAT представлена в таблице \ref{table:maxsat-length-research}.

\begin{table}[ht]
 \caption{История развития алгоритмов для задачи MAXSAT}
 \centering
 \begin{tabular}{|c|c|c|c|}
  \hline
  \textbf{Работа} & \textbf{Год} & \textbf{Результат} & \textbf{Δ} \\
  \hline
  Нидермайер и Россманит \cite{niedermeier1999new} & \citeyear{niedermeier1999new} & $O^*(1.1279^L) = O^*(2^{0.1737L})$ & -- \\
  Банзал и Раман \cite{bansal99} & \citeyear{bansal99} & $O^*(1.1057^L) = O^*(2^{0.1450L})$ & 16.5\% \\
  \hline
 \end{tabular}
 \label{table:maxsat-length-research}
\end{table}

Видно, что со времён работы \cite{bansal99} значимого улучшения не произошло. Причиной тому является недостаточное развитие смежного алгоритма для $(n,3)$-MAXSAT.

Как будет продемонстрированно в данном обзоре чуть ниже, простые правила сокращения позволяют убрать из формулы 1- и 2-перменные, таким образом. можно считать, что все переменные в $(n,3)$-MAXSAT-формуле являются 3-переменными, а длина формулы, таким образом, равняется утроенному количеству переменных. Одним из направлений исследований алгоритмов для задачи $(n,3)$-MAXSAT является исследование алгоритмов, экспоненциальных относительно количества переменных в формуле. В силу равненства $L = 3n$ таковой алгоритм автоматически является и алгоритмом относительно длины. История этих алгоритмов приведена в таблице \label{table:n3-maxsat-research}.

\begin{table}[ht]
 \caption{История развития алгоритмов для задачи $(n,3)$-MAXSAT}
 \centering
 \begin{tabular}{|c|c|c|}
  \hline
  \textbf{Работа} & \textbf{Год} & \textbf{Результат} \\
  \hline
  Раман, Равикумар и Рао \cite{raman1998simplified} & \citeyear{raman1998simplified} & $O^*(1.732^n)$ \\
  Банзал и Раман \cite{bansal99} & \citeyear{bansal99} & $O^*(1.3248^n)$ \\
  Куликов и Куцков \cite{kulikov2009new} & \citeyear{kulikov2009new} & $O^*(1.2721^n)$ \\
  Близнец \cite{bliznets2013new} & \citeyear{bliznets2013new} & $O^*(1.2600^n)$ \\
  Чэнь, Сюй и Ван \cite{chen15} & \citeyear{chen15} & $O^*(1.237^n)$ \\
  Ли, Cюй, Ван и Ян \cite{li2017improved} & \citeyear{li2017improved} & $O^*(1.194^n)$ \\
  Белова и Близнец \cite{belova18} & \citeyear{belova18} & $O^*(1.191^n)$ \\
  \hline
 \end{tabular}
 \label{table:n3-maxsat-research}
\end{table}

Отметим, что для достижения заявленной нами асимптотики с помощью введённого нами параметра, необходимо существование алгоритма для $(n,3)$-MAXSAT, работающего не хуже, чем за время $O^*(1.194^n)$.

В свою очередь, активное развитие алгоритмов для $(n,3)$-MAXSAT стало возможным благодаря представленной Близнецом и Головнёвым \cite{bliznets12} идеи сведения задачи к задаче о покрытии множества. Эта идея первоначально была применена к задаче о максимальной разрешимости, параметризованной ответом. История работ, связанных с этой формулировкой задачи, приведены в таблице \ref{table:maxsat-answer-research}.

\begin{table}[ht]
 \caption{Развитие алгоритмов для MAXSAT относительно ответа}
 \centering
 \begin{tabular}{|c|c|c|c|}
  \hline
  \textbf{Работа} & \textbf{Год} & \textbf{Результат} & \textbf{Δ} \\
  \hline
  Махаджан и Раман \cite{mahajan1999parameterizing} & \citeyear{mahajan1999parameterizing} & $O^*(1.618^k) = O^*(2^{0.695k})$ & -- \\
  Нидермайер и Россманит \cite{niedermeier1999new} & \citeyear{niedermeier1999new} & $O^*(1.400^k) = O^*(2^{0.486k})$ & 30\% \\
  Банзал и Раман \cite{bansal99} & \citeyear{bansal99} & $O^*(1.381^k) = O^*(2^{0.466k})$ & 4\% \\
  Чэнь и Кандж \cite{chen2004improved} & \citeyear{chen2004improved} & $O^*(1.370^k) = O^*(2^{0.455k})$ & 2.5\% \\
  Близнец и Головнёв \cite{bliznets12} & \citeyear{bliznets12} & $O^*(1.358^k) = O^*(2^{0.442k})$ & 2.8\% \\
  Чень и Сунь \cite{chen15} & \citeyear{chen15} & $O^*(1.325^k) = O^*(2^{0.406^k})$ & 8\% \\
  \hline
 \end{tabular}
 \label{table:maxsat-answer-research}
\end{table}

Отдельно хочется отметить развитие алгоритмов для задачи о максимальной разрешимости, параметризованной общим количеством дизъюнктов. Лучший результат в этой области $O^*(1.2989^m)$ был получен Сунь и др. \cite{xu19} в \citeyear{xu19} году. В то время как эта работа также использует идею сведения к задаче о покрытии множества, там также введено большое количество новых правил редукции. В то время как большая их часть увеличивает длину формулы (и, следовательно, не может быть применена к рассматриваемой задаче), одно из них оказывается весьма полезным. Подробнее об этом рассказано в данном обзоре ниже.
