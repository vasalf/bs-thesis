\newgeometry{top=2cm,bottom=2cm,left=3cm,right=1.5cm,nohead,includeheadfoot}

\section{Разбор 4-переменных}
\label{sec:n4}

\subsection{Общие наблюдения}
\label{subsec:n4:observations}

\firstpar{}Прежде всего, без ограничения общности, как и в прошлых главах, все 4-переменные являются либо $(2,2)$-переменными, либо $(3,1)$-переменными.

Мера $d$, как указано в мотивации в разделе \ref{subsec:measure:motivation}, разрабатывалась под задачу $(n,4)$-MAXSAT для сглаживания разницы между 3-переменными и 4-переменными. Для этой задачи она, как уже указывалось выше, имеет очень простой вид

$$
 d = 2n_3 + 4n_4
$$

Прежде всего, очевидно, что это число всегда чётное. Это объясняет то, что в этой главе все векторы расщепления чётные.

Более того, очень часто будет достаточно следующей леммы.

\begin{lemma}
 Пусть $l$ -- 4-литерал в формуле $(n,4)$-MAXSAT вида

 $$
  (l \vee C_1) \wedge \dots \wedge (l \vee C_i) \wedge (\ovl{l} \vee D_1) \wedge \dots \wedge (\ovl{l} \vee D_j) \wedge F'
 $$

 Пусть в объединении $C_i$ встречаются литералы хотя бы $k$ различных переменных.
 Тогда при означивании $l = 1$ мера $d$ уменьшается хотя бы на $4 + 2k$.
 \label{lemma:n4:one-side}
\end{lemma}

\begin{proof}
 В силу определения меры $d$ и правил упрощения, уничтожающих $2^-$-переменные, при удалении хотя бы одного литера любой переменной её вес в мере $d$ уменьшается хотя бы на 2. Также, 4-переменная $x$ элиминируется полностью. Таким образом, суммарное изменение весов переменных не меньше, чем $4 + 2k$.
\end{proof}

Это можно расширить до леммы про расщепление по переменной.

\begin{lemma}
 Пусть $x$ -- 4-переменная в формуле $(n,4)$-MAXSAT вида

 $$
  (x \vee C_1) \wedge \dots \wedge (x \vee C_i) \wedge (\ovl{x} \vee D_1) \wedge \dots \wedge (\ovl{x} \vee D_j) \wedge F'
 $$

 Пусть в объединении $C_i$ встречаются литералы хотя бы $k$ различных переменных, а в объединении $D_i$ встречаются литералы хотя бы $t$ различных переменных.
 Тогда при расщеплении по $x$ получается вектор расщепления не хуже, чем $(4 + 2k, 4 + 2t)$.
 \label{lemma:n4:branch-on}
\end{lemma}

\begin{proof}
 Применим дважды лемму \ref{lemma:n4:one-side}: к $x$ и к $\ovl{x}$.
\end{proof}

В некоторых случаях также имеет смысл обратить внимание на 4-переменные, встречающиеся дважды: у них вес изменяется на 4.

Далее представлен разбор случаев 4-переменных: в начале $(2,2)$-переменные, затем $(3,1)$-переменные -- не одиночки, затем $(3,1)$-одиночки.

\subsection{Разбор $(2,2)$-переменных}
\label{subsec:n4:22}

\firstpar{}В общем случае $(2,2)$-переменная обозначается следующим образом:

$$
 (x \vee C_1) \wedge (x \vee C_2) \wedge (\ovl{x} \vee D_1) \wedge (\ovl{x} \vee D_2) \wedge F'
$$

По правилу упрощения \ref{rrule:common:unit-clauses}, не более чем один из четырёх дизъюнктов $C_1$, $C_2$, $D_1$ и $D_2$ может быть пустым. Без ограничения общности, будем считать, что пустым может быть лишь $D_2$.

Наша цель -- показать, что у $(2,2)$-переменной обязательно должно быть много соседей.
Для начала предположим, что в $C_1$ и $C_2$ содержатся литералы лишь одной переменной.
Естественно, тогда длина каждого из $C_1$  и $C_2$ равна единице, а литералы в них равны либо противоположны.
Но если они противоположны, применимо правило \ref{rrule:common:almost-common}.
Для случая же если они равны, введём следующее правило упрощения.

\begin{rrule}
 Пусть $x$ -- $(2,2)$-переменная в обозначениях выше, а $y$ -- такая переменная, что $C_1 = C_2 = y$. Тогда можно назначить $y = \ovl{x}$.
\end{rrule}

\begin{proof}
 Для начала покажем, что $y = 0 \Rightarrow x = 1$.

 И правда, если означить $y = 0$, то $x$ останется $4^-$-переменной с как двумя положительными вхождениями в дизъюнкт длины 1, а значит, по правилу упрощения \ref{rrule:common:unit-clauses} этой переменной назначается значение 1.

 Теперь покажем, что $y = 1 \Rightarrow x = 0$.

 И правда, если означить $y = 1$, то $x$ остаётся $(0,2^-)$-переменной, и по правилу \ref{rrule:common:i0} тогда этой переменной назначается значение 0.
\end{proof}

Хотя это и не имеет решающего значения для алгоритма, всё же подчеркнём, что формула остаётся экземпляром задачи $(n,4)$-MAXSAT: как минимум четыре вхождения новой $8^-$-переменной мгновенно уничтожаются правилом \ref{rrule:common:complementary}.

Таким образом, если указанные правила упрощения неприменимы, в объединении $C_1$ и $C_2$ есть литералы хотя бы двух различных переменных.
Более того, это точно также верно и для $D_1$ и $D_2$, если оба они непусты.
Это позволяет сразу воспользоваться леммой \ref{lemma:n4:branch-on} и вывести следующее правило.

\begin{brule}
 Пусть $x$ -- $(2,2)$-переменная в обозначениях выше, не входящая в дизъюнкты длины 1. Тогда расщепиться по $x$.

 Это даёт хотя бы $(8,8)$-расщепление.
 \label{brule:n4:22:nouc}
\end{brule}

\begin{proof}
 Выше показано, что и в объединении $C_1$ и $C_2$, и в объединении $D_1$ и $D_2$ содержатся литералы по меньшей мере двух различных переменных. Таким образом, по лемме \ref{lemma:n4:branch-on}, это хотя бы $(8,8)$-расщепление.
\end{proof}

С этого места и ниже в этом разделе разбирается случай $(2,2)$-переменных, входящих в дизъюнкт длины 1.
Ниже выведено хорошее правило расщепление и для этого случая, но предварительно необходимо разобрать один частный случай: когда в $D_1$ находится единственный литерал 3-переменной, входящей в $C_i$. Для этого вводятся следующие правила упрощения.

\begin{rrule}
 Пусть $x$ -- $(2,2)$-переменная в обозначениях выше, входящая в дизъюнкт длины 1. Если $C_1$ и $C_2$ содержат противоположные литералы, можно означить $x = 0$.
 \label{rrule:n4:22:uc-compl}
\end{rrule}

\begin{proof}
 В таком случае хотя бы один из дизъюнктов $C_1$ и $C_2$ выполнен в любом означивании.

 Рассмотрим оптимальное означивание и пусть там $x = 1$. Тогда из дизъюнктов с $x$ это означивание выполняет не более трёх. Если назначить в этом означивании $x = 1$, выполнятся из них тоже хотя бы три, а выполнимость других дизъюнктов не изменится. Значит, количество выполненных дизъюнктов не уменьшилось. Таким образом, существует оптимальное означивание с $x = 0$.
\end{proof}

В частности, упомянутая 3-переменная может входить в дизъюнкты $C_1$ или $C_2$ лишь однажды: одинаковые вхождения запрещены правилом \ref{rrule:common:xu19rr9}, а различные -- правилом \ref{rrule:n4:22:uc-compl}.

\begin{rrule}
 Пусть $x$ -- $(2,2)$-переменная в обозначениях выше, входящая в дизъюнкт длины 1. Пусть $|D| = 1$. Обозначим $D = l$. Если $l$ -- литерал 3-переменной, и литерал $l$ входит в $C_1$ или $C_2$, можно означить $x = 0$. 
 \label{rrule:n4:22:uc-3v-pos-c}
\end{rrule}

\begin{proof}
 Не умаляя общности, пусть $C_1 = l \vee C_1'$.

 Рассмотрим два случая на значение $l$ и покажем, что в каждом из них означивание $x = 0$ оптимально.

 Если $l = 0$, остаётся формула следующего вида:

 $$
  (x \vee C_1') \wedge (x \vee C_2) \wedge \ovl{x} \wedge \ovl{x} \wedge F'
 $$

 Тут правило упрощения \ref{rrule:common:unit-clauses} разрешает означить $x = 0$.

 Если $l = 1$, остаётся формула следующего вида:

 $$
  (x \vee C_2) \wedge \ovl x \wedge F'
 $$

 Тут то же правило упрощения \ref{rrule:common:unit-clauses} разрешает означить $x = 0$.
\end{proof}

\begin{rrule}
 Пусть $x$ -- $(2,2)$-переменная в обозначениях выше, входящая в дизъюнкт длины 1. Пусть $|D| = 1$. Обозначим $D = l$. Если $l$ -- литерал 3-переменной, и литерал $\ovl{l}$ входит в $C_1$ или $C_2$, и литерал $\ovl{l}$ встречается в $F'$, можно назначить $x = l$.
 \label{rrule:n4:22:uc-3v-neg-c-neg-f}
\end{rrule}

\begin{proof}
 Не умаляя общности, пусть $C_1 = \ovl{l} \vee C_1'$.
 Обозначим также $F' = (\ovl{l} \vee E) \wedge F''$.

 Рассмотрим два случая на значение $x$.

 Если $x = 0$, остаётся формула следующего вида:

 $$
  (\ovl{l} \vee E) \wedge (\ovl{l} \vee C_1') \wedge C_2 \wedge F''
 $$

 Тогда правило упрощения \ref{rrule:common:i0} означивает $l = 0$.

 Если же $x = 1$, остаётся формула следующего вида:

 $$
  (\ovl{l} \vee E) \wedge l \wedge F'' 
 $$

 Тогда правило упрощения \ref{rrule:common:unit-clauses} означивает $l = 1$.
 \label{rrule:n4:22:uc-3v-neg-c-pos-f}
\end{proof}

Отметим, что после применения этого правила четыре литерала новой 7-переменной мгновенно элиминируются правилом \ref{rrule:common:complementary}, а значит, формула остаётся экземпляром задачи $(n,4)$-MAXSAT.

\begin{rrule}
 Пусть $x$ -- $(2,2)$-переменная в обозначениях выше, входящая в дизъюнкт длины 1. Пусть $|D| = 1$. Обозначим $D = l$. Если $l$ -- литерал 3-переменной, и литерал $\ovl{l}$ входит в $C_1$ или $C_2$, и литерал $l$ встречается в $F'$, можно применить следующее правило упрощения:

 \begin{gather*}
  (l \vee E) \wedge (x \vee \ovl{l} \vee C_1') \wedge (x \vee C_2) \wedge (\ovl{x} \vee l) \wedge \ovl{x} \wedge F'' \\
  \rightarrow
  (x \vee C_1' \vee E) \wedge (x \vee C_2) \wedge \ovl{x} \wedge F''
 \end{gather*}
 {\color{white} А вот это прикольно.}
\end{rrule}

\begin{proof}
 Из пяти дизъюнктов в левой части четыре можно выполнить при любых значениях $E$, $C_1'$ и $C_2$, означив $x = l = 1$. Пять же выполнимы только если выполнены $C_2$ и хотя бы один из $E$ и $C_1'$.

 Абсолютно такое же рассуждение верно и про правую часть, только при этом количество выполненных дизъюнктов в каждом из двух случаев меньше на 2.
\end{proof}

После того, как предыдущие правила упрощения неприменимы, понятно, что если $l$ и является 3-переменной, то во всяком случае не встречающейся в объединении $C_1$ и $C_2$. Тогда применим следующее правило расщепления:

\begin{brule}
 Пусть $x$ -- $(2,2)$-переменная в обозначениях выше, входящая в дизъюнкт длины 1. Тогда расщепиться на два случая:
 \begin{enumerate}
  \item $x = 0$
  \item $x = 1$, и, если $|D| = 1$, $D = 1$
 \end{enumerate}

 Это даёт хотя бы $(6,10)$-расщепление.
 \label{brule:n4:22:uc}
\end{brule}

\begin{proof}
 Докажем корректность.

 Пусть в оптимальном означивании $x = 1$ и $D = 0$. Тогда из четырёх дизъюнктов с $x$ выполнено ровно 2. Означив $x = 0$, мы выполним хотя бы 2, не изменив выполненность других дизъюнктов. Таким образом, существует оптимальное означивание, где либо $x = 0$, либо $x = 1$ и $D = 1$.

 Докажем теперь вектор.

 Если $|D| > 1$, то по лемме \ref{lemma:n4:branch-on} это хотя бы $(8,8)$-расщепление: в объединении $C_1$ и $C_2$ есть хотя бы две различные переменные, и в $D$ есть хотя бы две различные переменные.

 Если $|D| = 1$, обозначим $D = l$.

 Если $l$ -- литерал 4-переменной, то это хотя бы $(8,8)$-расщепление: во втором случае, так как в объединении $C_1$ и $C_2$ есть хотя бы две переменные, по лемме \ref{lemma:n4:one-side} мера $d$ уменьшается хотя бы на 8, а в первом случае исчезают 4-переменные $x$ и $l$ и таким образом мера также уменьшается хотя бы на 8.

 Если $l$ -- литерал 3-переменной, и предыдущие правила неприменимы, то эта переменная не входит в объединение $C_j$. Тогда в первом случае мера по лемме \ref{lemma:n4:one-side} уменьшается хотя бы на 6, а во втором случае исчезают 4-переменная $x$, 3-переменная $l$ и литералы ещё хотя бы двух переменных в $C_1$ и $C_2$, таким образом, мера уменьшается хотя бы на 10.

 Так как $(6,10)$ -- худший из представленных векторов, он и является оценкой на худший случай.
\end{proof}

Это правило завершает разбор $(2,2)$-переменных.

\subsection{Разбор $(3,1)$-переменных -- не одиночек}
\label{subsec:n4:31-ns}

$(3,1)$-переменные обозначаются так:

$$
 (x \vee C_1) \wedge (x \vee C_2) \wedge (x \vee C_3) \wedge (\ovl{x} \vee D) \wedge F'
$$

Напомним, $(3,1)$-переменная называется одиночкой, если $D$ пусто.
В данном разделе мы рассматриваем случай непустого $D$.
В этом случае оказывается достаточно применить лемму \ref{lemma:n5:i1}.

\begin{brule}
 Пусть $x$ -- $(3,1)$-переменная, не являющаяся одиночкой. 
 Тогда расщепиться по лемме \ref{lemma:n5:i1}.

 Это даёт хотя бы $(6,10)$-расщепление.
 {\color{white} Это тоже грязь.}
 \label{brule:n4:31:ns}
\end{brule}

\begin{proof}
 Во-первых, отметим, что в объединении $C_i$ есть хотя бы две различные переменные.
 Если бы была только одна, все литералы не могли бы быть одинаковыми по правилу \ref{rrule:common:xu19rr9}, а пара противоположных не могла бы туда входить по правилу \ref{rrule:common:almost-common}.

 Далее, если $|D| \geq 2$, то это хотя бы $(8,8)$-расщепление. В первом случае по лемме \ref{lemma:n4:one-side} длина уменьшается хотя бы на 8, а во втором случае мы означиваем $x$ и ещё хотя бы две $3^+$-переменные.

 Иначе, $|D| = 1$. Обозначим тогда $D = l$.

 Если $l$ -- литерал 4-переменной, то это тоже хотя бы $(8,8)$-расщепление: как и в случае $|D| \geq 2$, в первом случае мера уменьшается хотя бы на 8, а во втором случае мы означиваем 4-переменные $x$ и $l$.

 Иначе, $l$ -- литерал 3-переменной.

 Если в объединении $C_i$ содержатся литералы хотя бы трёх различных переменных, то это хотя бы $(6,10)$-расщепление: в первом случае по лемме \ref{lemma:n4:one-side} мера уменьшается хотя бы на 10, а во втором мы означиваются 4-переменная $x$ и 3-переменная $l$.

 Иначе, в объединении $C_i$ есть переменная, встречающаяся там хотя бы дважды.
 Обозначим её за $y$.

 Если $y$ -- 4-переменная, то это тоже хотя бы $(6,10)$-расщепление: в первом случае означивается 4-переменная $x$, полностью исчезает $4$-переменная $y$ как $2^-$-переменная и исчезают также литералы ещё как минимум одной $3^+$-переменной, а во втором случае означиваются 4-переменная $x$ и 3-переменная $l$.

 Иначе, $y$ -- 3-переменная. Поскольку правило \ref{rrule:common:xu19rr9} неприменимо, вхождения $y$ в $C_i$ должны быть разнознаковыми. Без ограничения общности положим $C_1 = y \vee C_1'$ и $C_2 = \ovl{y} \vee C_2'$.

 Поскольку в объединении $C_i$ есть литералы ровно двух различных переменных, $C_3 = m$, где $m$ -- литерал переменной, не совпадающей с $x$ и $y$. Однако $m$ и $l$ могут быть литералами одной и той же переменной.

 Если это так, то хотя бы один из дизъюнктов $C_1'$ и $C_2'$ должен быть пустым (так как $l$ -- литерал 3-переменной), и тогда это хотя бы $(8,8)$-расщепление: в первом случае мера по лемме \ref{lemma:n4:one-side} уменьшается хотя бы на 8, а во втором означиваются 4-переменная $x$ и 3-переменные $y$ и $l$.

 Если же $l$ и $m$ -- литералы различных переменных, то это тоже хотя бы $(8,8)$-расщепление: в первом случае мера по лемме \ref{lemma:n4:one-side} уменьшается хотя бы на 8, а во-втором означиваются 4-переменная $x$ и 3-переменные $l$ и $m$.

 Так как $(6,10)$ -- худший из представленных векторов, он и является оценкой на худший случай.
\end{proof}

Это правило является достаточным для $(3,1)$-переменных, не являющихся одиночками.

\subsection{Разбор $(3,1)$-одиночек}
\label{subsec:n4:31-2c}

Теперь допустим, что $|D| = 0$, но хотя бы для одного $i$ верно $|C_i| = 1$.
Не умаляя общности, обозначим $C_1 = l$.
Оказывается, что тогда леммы \ref{lemma:n5:i1} тоже достаточно.

\begin{brule}
 Пусть $x$ -- $(3,1)$-одиночка, и при этом $|C_1| = 1$.
 Тогда расщепиться по лемме \ref{lemma:n5:i1}.

 Это даёт хотя бы $(6,10)$-расщепление.
 \label{brule:n4:31:2c}
\end{brule}

\begin{proof}
 Если $l$ -- литерал 4-переменной, то это хотя бы $(8,8)$-расщепление: в первом случае по лемме \ref{lemma:n4:one-side} мера уменьшается хотя бы на 8, а во втором случае означиваются 4-переменные $x$ и $l$.

 Иначе, $l$ -- литерал 3-переменной.

 Если в объединении $C_i$ есть литералы хотя бы трёх различных переменных, это хотя бы $(6,10)$-расщепление: в первом случае по лемме \ref{lemma:n4:one-side} мера уменьшается хотя бы на 10, а во втором случае означиваются 4-переменная $x$ и 3-переменная $l$.

 Иначе, в объединении $C_i$ есть литералы лишь двух различных переменных.
 Значит, либо для какого-то $j \neq 1$ выполняется $|C_j| = 1$, либо 3-переменная $l$ входит в это объединение трижды. Последнее запрещено правилом \ref{rrule:common:xu19rr9}, значит, для какого-то $j$ выполняется $|C_j| = 1$.

 Тогда это хотя бы $(8,8)$-расщепление: в первом случае по лемме \ref{lemma:n4:one-side} мера уменьшается хотя бы на 8, а во втором случае означиваются 4-переменная $x$, 3-переменная $l$ и ещё одна $3^+$-переменная.

 Так как $(6,10)$ -- худший из представленных векторов, он и является оценкой на худший случай.
\end{proof}

Остался неразобранным случай, когда $|D| = 0$ и для любого $i$ выполняется $|C_i| \geq 2$.
В таком случае лемма \ref{lemma:n5:i1} сводится к расщеплению по $x$.

Задачей этого раздела является нахождение большого количества соседей у таких переменных.
Для этого предлагается разобрать случаи, когда одна и та же переменная входит в объединение $C_i$ несколько раз.
Тогда, так как $\sum |C_i| \geq 6$, и должен остаться только случай с большим количеством соседей.

Для начала заметим, что никакая переменная на текущий момент не может входить в объединение $C_i$ трижды. И правда, для 3-переменной это запрещено правилом упрощения \ref{rrule:common:xu19rr9}. Поскольку к $(2,2)$-переменным применимы правила из раздела \ref{subsec:n4:22}, их в формуле не осталось. Три положительных вхождения $(3,1)$-переменной также запрещены правилом упрощения \ref{rrule:common:xu19rr9}. Если же у $(3,1)$-переменной есть отрицательное вхождение в объединение $C_i$, то она не является одиночкой и к ней применимы правила из раздела \ref{subsec:n4:31-ns}.

Далее представлены правила, работающие со случаями, когда переменные входят в объединение $C_i$ дважды.

\begin{rrule}
 Если $x$ -- $(3,1)$-одиночка в обозначениях выше, причём длины всех $C_i$ не меньше двух, а $y$ -- $(2,1)$-переменная, то можно убрать $x$ из дизъюнкта с положительным вхождением $y$, то есть

 \begin{gather*}
  (y \vee E) \wedge (x \vee y \vee C_1') \wedge (x \vee \ovl{y} \vee C_2') \wedge (x \vee C_3) \wedge \ovl{x} \wedge F''\\
  \rightarrow
  (y \vee E) \wedge (y \vee C_1') \wedge (x \vee \ovl{y} \vee C_2') \wedge (x \vee C_3) \wedge \ovl{x} \wedge F''
 \end{gather*}
 {\color{white} И вот это тоже прикольно.}
 \label{rrule:n4:31:3v-2}
\end{rrule}

\begin{proof}
 Очевидно, любое означивание переменных в левой части правила выполняет не меньше дизъюнктов, чем в правой части, так как правая часть получается из левой вычёркиванием литералов.
 Покажем теперь, что существует оптимальное для левой части означивание, выполняющее столько же дизъюнктов в правой части, сколько и в левой. Тогда естественным образом окажется, что ответы на задачу MAXSAT для двух частей правила совпадают.

 Рассмотрим какое-то оптимальное означивание переменных в левой части правила.

 Очевидно, если $C_1' = 1$, то в левой и в правой части правила количество выполненных дизъюнктов одинаково.

 Если же $C_1' = 0$, то эти количества могут быть различными только при $x = 1$ и $y = 0$ (когда в левой части выполнено больше дизъюнктов). Но в таком случае при назначении $y = 1$ в таком означивании количество выполненных дизъюнктов не уменьшается (то есть означивание остаётся оптимальным для левой части), а в правой части становится таким же, как в левой.
 Такое означивание и требовалось найти.
\end{proof}

Для 4-переменных же универсального правила найти не удаётся, однако можно ограничить число таких переменных с двумя вхождениями в объединение $C_i$.

\begin{brule}
 Пусть $x$ -- $(3,1)$-одиночка в обозначениях выше, и длины всех $C_i$ не меньше двух. Если в объединении $C_i$ встречаются две пары литералов двух 4-переменных, расщепиться по $x$.

 Это даёт хотя бы $(4,14)$-расщепление.
 \label{brule:n4:31:4v-2}
\end{brule}

\begin{proof}
 В случае $x = 0$ исчезает 4-переменная $x$.

 В случае $x = 1$ означиваются 4-переменная $x$, две другие 4-переменные (так как они становятся $2^-$-переменными, а также исчезают литералы ещё хотя бы одной другой $3^+$-переменной, так как $\sum |C_i| \geq 6$, а указанные в условии 4-переменные занимают лишь четыре из этих как минимум шести литералов.
\end{proof}

Наконец, на текущий момент можно утверждать, что у $x$ есть хотя бы пять соседей: $\sum |C_i| \geq 6$ и лишь одна пара из этих литералов может быть занята одной и той же переменной, причём это должна быть 4-переменная.
Значит, можно воспользоваться леммой \ref{lemma:n4:branch-on}.

\begin{brule}
 Пусть $x$ -- $(3,1)$-переменная в обозначениях выше, и предыдущие правила к ней неприменимы.
 Тогда расщепиться по $x$.

 Это даёт хотя бы $(4,14)$-расщепление.
 \label{brule:n4:31:final}
\end{brule}

\begin{proof}
 Это применение леммы \ref{lemma:n4:branch-on}: в объединении $C_i$ есть хотя бы пять различных переменных.
\end{proof}

Это правило завершает разбор 4-переменных.

\subsection{Выводы}
\label{subsec:n4:summary}

\begin{itemize}
 \item Продемонстрированы свойства меры $d$ при разборе $4^-$-переменных.
 \item На основании этих свойств представлены правила, элиминирующие все 4-переменные в формуле, с худшим вектором расщепления $(6,10)$.
\end{itemize}

% vim: spell spelllang=ru_yo,en_gb
