\newgeometry{top=2cm,bottom=2cm,left=3cm,right=1.5cm,nohead,includeheadfoot}

\section{Разбор 4-переменных}
\label{sec:n4}

\subsection{Общие наблюдения}
\label{subsec:n4:observations}

\firstpar{}Прежде всего, без ограничения общности, как и в прошлых главах, все 4-переменные являются либо $(2,2)$-переменными, либо $(3,1)$-переменными.

Мера $d$, как указано в мотивации в разделе \ref{subsec:measure:motivation}, разрабатывалась под задачу $(n,4)$-MAXSAT для сглаживания разницы между 3-переменными и 4-переменными. Для этой задачи она, как уже указывалось выше, имеет очень простой вид

$$
 d = 2n_3 + 4n_4
$$

Прежде всего, очевидно, что это число всегда чётное. Это объясняет то, что в этой главе все векторы расщепления чётные.

Более того, очень часто будет достаточно следующей леммы.

\begin{lemma}
 Пусть $l$ -- 4-литерал в формуле $(n,4)$-MAXSAT вида

 $$
  (l \vee C_1) \wedge \dots \wedge (l \vee C_i) \wedge (\ovl{l} \vee D_1) \wedge \dots \wedge (\ovl{l} \vee D_j) \wedge F'
 $$

 Пусть в объединении $C_i$ встречаются литералы хотя бы $k$ различных переменных.
 Тогда при означивании $l = 1$ мера $d$ уменьшается хотя бы на $4 + 2k$.
 \label{lemma:n4:one-side}
\end{lemma}

\begin{proof}
 В силу определения меры $d$ и правил упрощения, уничтожающих $2^-$-переменные, при удалении хотя бы одного литера любой переменной её вес в мере $d$ уменьшается хотя бы на 2. Также, 4-переменная $x$ элиминируется полностью. Таким образом, суммарное изменение весов переменных не меньше, чем $4 + 2k$.
\end{proof}

Это можно расширить до леммы про расщепление по переменной.

\begin{lemma}
 Пусть $x$ -- 4-переменная в формуле $(n,4)$-MAXSAT вида

 $$
  (x \vee C_1) \wedge \dots \wedge (x \vee C_i) \wedge (\ovl{x} \vee D_1) \wedge \dots \wedge (\ovl{x} \vee D_j) \wedge F'
 $$

 Пусть в объединении $C_i$ встречаются литералы хотя бы $k$ различных переменных, а в объединении $D_i$ встречаются литералы хотя бы $t$ различных переменных.
 Тогда при расщеплении по $x$ получается вектор расщепления не хуже, чем $(4 + 2k, 4 + 2t)$.
 \label{lemma:n4:branch-on}
\end{lemma}

\begin{proof}
 Применим дважды лемму \ref{lemma:n4:one-side}: к $x$ и к $\ovl{x}$.
\end{proof}

В некоторых случаях также имеет смысл обратить внимание на 4-переменные, встречающиеся дважды: у них вес изменяется на 4.

Далее представлен разбор случаев 4-переменных: в начале $(2,2)$-переменные, затем $(3,1)$-переменные -- не одиночки, затем $(3,1)$-одиночки.

\subsection{Разбор $(2,2)$-переменных}
\label{subsec:n4:22}

\firstpar{}В общем случае $(2,2)$-переменная обозначается следующим образом:

$$
 (x \vee C_1) \wedge (x \vee C_2) \wedge (\ovl{x} \vee D_1) \wedge (\ovl{x} \vee D_2) \wedge F'
$$

По правилу упрощения \ref{rrule:common:unit-clauses}, не более чем один из четырёх дизъюнктов $C_1$, $C_2$, $D_1$ и $D_2$ может быть пустым. Без ограничения общности, будем считать, что пустым может быть лишь $D_2$.

Наша цель -- показать, что у $(2,2)$-переменной обязательно должно быть много соседей.
Для начала предположим, что в $C_1$ и $C_2$ содержатся литералы лишь одной переменной.
Естественно, тогда длина каждого из $C_1$  и $C_2$ равна единице, а литералы в них равны либо противоположны.
Но если они противоположны, применимо правило \ref{rrule:common:almost-common}.
Для случая же если они равны, введём следующее правило упрощения.

\begin{rrule}
 Пусть $x$ -- $(2,2)$-переменная в обозначениях выше, а $y$ -- такая переменная, что $C_1 = C_2 = y$. Тогда можно назначить $y = \ovl{x}$.
\end{rrule}

\begin{proof}
 Для начала покажем, что $y = 0 \Rightarrow x = 1$.

 И правда, если означить $y = 0$, то $x$ останется $4^-$-переменной с как двумя положительными вхождениями в дизъюнкт длины 1, а значит, по правилу упрощения \ref{rrule:common:unit-clauses} этой переменной назначается значение 1.

 Теперь покажем, что $y = 1 \Rightarrow x = 0$.

 И правда, если означить $y = 1$, то $x$ остаётся $(0,2^-)$-переменной, и по правилу \ref{rrule:common:i0} тогда этой переменной назначается значение 0.
\end{proof}

Хотя это и не имеет решающего значения для алгоритма, всё же подчеркнём, что формула остаётся экземпляром задачи $(n,4)$-MAXSAT: как минимум четыре вхождения новой $8^-$-переменной мгновенно уничтожаются правилом \ref{rrule:common:complementary}.

Таким образом, если указанные правила упрощения неприменимы, в объединении $C_1$ и $C_2$ есть литералы хотя бы двух различных переменных.
Более того, это точно также верно и для $D_1$ и $D_2$, если оба они непусты.
Это позволяет сразу воспользоваться леммой \ref{lemma:n4:branch-on} и вывести следующее правило.

\begin{brule}
 Пусть $x$ -- $(2,2)$-переменная в обозначениях выше, не входящая в дизъюнкты длины 1. Тогда расщепиться по $x$.

 Это даёт хотя бы $(8,8)$-расщепление.
 \label{brule:n4:22:nouc}
\end{brule}

\begin{proof}
 Выше показано, что и в объединении $C_1$ и $C_2$, и в объединении $D_1$ и $D_2$ содержатся литералы по меньшей мере двух различных переменных. Таким образом, по лемме \ref{lemma:n4:branch-on}, это хотя бы $(8,8)$-расщепление.
\end{proof}

С этого места и ниже в этом разделе разбирается случай $(2,2)$-переменных, входящих в дизъюнкт длины 1.

Для остальных случаев 

% vim: spell spelllang=ru_yo,en_gb
