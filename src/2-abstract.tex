\newgeometry{top=2cm,bottom=2cm,left=3cm,right=1.5cm,nohead,includeheadfoot}

\section*{Аннотация}
\label{sec:annotation}
\addcontentsline{toc}{section}{\nameref{sec:annotation}}

\firstpar{}Задача булевой разрешимости -- исторически первая задача, для которой была доказана NP-полнота. Её оптимизационная версия, задача максимальной разрешимости, состоящая в выполнении наибольшего количества дизъюнктов в булевой формуле, также является NP-полной. Несмотря на то, что в предположении гипотезы об экспоненциальном времени эти задачи не могут быть решены за субэкспоненциальное время, задача максимальной разрешимости имеет большое количество применений, и подходы к этой задаче активно изучаются. В последние годы исследования версий задачи максимальной разрешимости, параметризованных общим количеством дизэюнктов и количеством выполненных дизъюнктов, сильно продвинулись за счёт введения сильных правил сокращения, основанных на правиле резолюции, и новых техник сведения экземпляра задачи к экземпляру задачи о покрытии множества. Другой важный результат заключаетсчя в том, что задача $(n,3)$-MAXSAT, параметризованная количеством переменных, решается гораздо быстрее, чем в общем случае \cite{belova18}. В данной работе рассматривается задача максимальной разрешимости, решаемая относительно длины формулы, то есть суммарного количества литералов во всех дизъюнктах. Несмотря на то, что некоторые новые правила оказываются полезными для такой задачи, большинство из них увеличивают длину формулы и не могут быть применены. В этой работе представлены новые правила сокращения, не увеличивающие длину формулы. Также предлагается новый параметр с пониженной стоимостью 3-переменных, использующий то, что $(n,3)$-MAXSAT решается гораздо быстрее, чем общий случай задачи максимальной разрешимости. Комбинация двух методов позволяет получить алгоритм, работающий за время $O^*(1.093^L)$. Это улучшает предыдущую верхнюю оценку в $O^*(1.106^L)$, полученную Банзалом и Раманом \cite{bansal99}.

\vspace{14pt}

\textit{Ключевые слова:} задача максимальной разрешимости, параметризованные алгоритмы, точные экспоненциальные алгоритмы

\newpage

\begin{otherlanguage}{english}
 Satisfiability problem is historically the first problem that was proven to be NP-complete. Maximum Satisfiability, being its optimization version, is also NP-complete. Though under assumption of Exponential Time Hypothesis those problems cannot be solved in subexponential time, Maximum Satisfiability have many applications, so approaches to solve its instances are nevertheless heavily studied. In the last years, research for versions of Maximum Satisfiability parameterized by the total number of clauses and the number of satisfied clauses have been pushed forward by introducing powerful resolution-based reduction rules and new techniques to reduce the problem instance to a Set Cover instance. In our work, we consider Maximum Satisfiability version parameterized by the formula length (sum of number of literals in each clause). Another important recent result is $(n,3)$-MAXSAT problem being solved in much better time than general case when parameterized by the number of variables \cite{belova18}. Though some of new reduction rules appear to be very useful in this parameterization, most of them increase the length of the formula, and hence cannot be used for solving the problem in this parameterization. In our work, we introduce new reduction rules that do not increase the formula length. Then, we decrease the parameter value to discount the 3-variables, given that $(n,3)$-MAXSAT can be solved in much better time than general MAXSAT. The combination of two techniques produces an algorithm with running time $O^∗(1.093^{|F|})$, improving the previous bound of $O^∗(1.106^{|F|})$ by Bansal and Raman \cite{bansal99}. 

\vspace{14pt}

\textit{Keywords:} Maximum Satisfiability, Parameterized Complexity, Exact Exponential Algorithms
\end{otherlanguage}

% vim: spell spelllang=ru_yo,en_gb
