\documentclass[a4paper,14pt]{extarticle}

\usepackage{amsmath}
\usepackage{amssymb}
\usepackage{amsthm}
\usepackage{fontspec}
\usepackage{polyglossia}
\usepackage[colorlinks=true,urlcolor=black,linkcolor=black,filecolor=black,citecolor=black]{hyperref}
\usepackage{geometry}
\usepackage[style=gost-numeric]{biblatex}
\usepackage{setspace}
\usepackage{lipsum}
\usepackage{caption}

\defaultfontfeatures{Ligatures={TeX}}
\setmainfont{CMU Serif}
\setsansfont{CMU Sans Serif}
\setmonofont{CMU Typewriter Text}

\setdefaultlanguage[spelling=modern]{russian}
\setotherlanguage[variant=british]{english}
\frenchspacing
\onehalfspacing

\addbibresource{main.bib}

\newcommand{\ovl}[1]{\overline{#1}}
% For some reason indentfirst doesn't work.
\newcommand{\firstpar}[1]{\hspace{1cm}}

\captionsetup[table]{justification=raggedleft,singlelinecheck=false}
\numberwithin{table}{section}

\theoremstyle{plain}
\newtheorem{rrule}{Правило сокращения}[section]
\newtheorem{brule}{Правило расщепления}[rrule]
\newtheorem{lemma}{Лемма}[section]
\newtheorem{theorem}{Теорема}[lemma]

\theoremstyle{remark}
\newtheorem*{note}{Замечание}

\title{Новый алгоритм для задачи выполнения наибольшего количества дизъюнктов}
\author{Алфёров Василий Викторович}

\begin{document}
 \def\ignore#1{}
\let\orignewcommand\newcommand
\let\origNeedsTeXFormat\NeedsTeXFormat

\let\newcommand\renewcommand
\makeatletter
\input{size12.clo}%
\makeatother  
\let\newcommand\orignewcommand

\makeatletter
\begin{titlepage}
 \newgeometry{top=2cm,bottom=2cm,left=2cm,right=1.5cm,nohead,includeheadfoot}
 \setlength{\parskip}{1em}
 \setlength{\parindent}{0pt}
 \setstretch{1.0}
 \begin{center}
  Федеральное государственное автономное образовательное учреждение \\
  высшего образования \\
  «Национальный исследовательский университет «Высшая школа экономики»

  Факультет Санкт-Петербургская школа физико-математических и компьютерных наук \\
  Департамент информатики

  Основная профессиональная образовательная программа \\
  «Прикладная математика и информатика»

  \vspace{3em}

  {\bfseries ВЫПУСКНАЯ КВАЛИФИКАЦИОННАЯ РАБОТА} \\
  на тему \\
  {\bfseries\@title}
 \end{center}

 \vspace{3em}

 \onehalfspacing

 Выполнил студент группы БПМ161, 4 курса, \\
 \@author

 \vspace{3em}

 Научный руководитель: \\
 Кандидат физико-математических наук, \\
 доцент департамента информатики, \\
 Близнец Иван Анатольевич

 \vfill

 \begin{center}
  Санкт-Петербург \\
  2020
 \end{center}
\end{titlepage}
\makeatother

\let\NeedsTeXFormat\ignore
\let\newcommand\renewcommand
\makeatletter
\input{size14.clo}%
\makeatother  
\let\newcommand\orignewcommand
\let\NeedsTeXFormat\origNeedsTeXFormat

\addtocounter{page}{1}


 \newgeometry{top=2cm,bottom=2cm,left=3cm,right=1.5cm,nohead,includeheadfoot}
\renewcommand{\contentsname}{Оглавление}
\tableofcontents


 \setlength{\parindent}{1cm}

 \newgeometry{top=2cm,bottom=2cm,left=3cm,right=1.5cm,nohead,includeheadfoot}

\section*{Аннотация}
\label{sec:annotation}
\addcontentsline{toc}{section}{\nameref{sec:annotation}}

\firstpar{}Задача булевой выполнимости -- исторически первая задача, для которой была доказана NP-полнота. Её оптимизационная версия, задача максимальной выполнимости, состоящая в выполнении наибольшего количества дизъюнктов в булевой формуле, также является NP-полной. Несмотря на то, что в предположении гипотезы об экспоненциальном времени эти задачи не могут быть решены за субэкспоненциальное время, задача максимальной выполнимости имеет большое количество применений, и подходы к этой задаче активно изучаются. В последние годы исследования версий задачи максимальной выполнимости, параметризованных общим количеством дизъюнктов и количеством выполненных дизъюнктов, сильно продвинулись за счёт введения сильных правил упрощения, основанных на правиле резолюции, и новых техник сведения экземпляра задачи к экземпляру задачи о покрытии множества. Другой важный результат заключается в том, что задача $(n,3)$-MAXSAT, параметризованная количеством переменных, решается гораздо быстрее, чем в общем случае \cite{belova18}. В данной работе рассматривается задача максимальной выполнимости, решаемая относительно длины формулы, то есть суммарного количества литералов во всех дизъюнктах. Несмотря на то, что некоторые новые правила оказываются полезными для такой задачи, большинство из них увеличивают длину формулы и не могут быть применены. В этой работе представлены новые правила упрощения, не увеличивающие длину формулы. Также предлагается новый параметр с пониженной стоимостью 3-переменных, использующий то, что $(n,3)$-MAXSAT решается гораздо быстрее, чем общий случай задачи максимальной выполнимости. Комбинация двух методов позволяет получить алгоритм, работающий за время $O^*(1.093^L)$. Это улучшает предыдущую верхнюю оценку в $O^*(1.106^L)$, полученную Банзалом и Раманом \cite{bansal99}.

\vspace{14pt}

\textit{Ключевые слова:} задача максимальной выполнимости, параметризованные алгоритмы, точные экспоненциальные алгоритмы

\newpage

\begin{otherlanguage}{english}
 Satisfiability problem is historically the first problem that was proven to be NP-complete. Maximum Satisfiability, being its optimization version, is also NP-complete. Though under assumption of Exponential Time Hypothesis those problems cannot be solved in subexponential time, Maximum Satisfiability have many applications, so approaches to solve its instances are nevertheless heavily studied. In the last years, research for versions of Maximum Satisfiability parameterized by the total number of clauses and the number of satisfied clauses have been pushed forward by introducing powerful resolution-based reduction rules and new techniques to reduce the problem instance to a Set Cover instance. In our work, we consider Maximum Satisfiability version parameterized by the formula length (sum of number of literals in each clause). Another important recent result is $(n,3)$-MAXSAT problem being solved in much better time than general case when parameterized by the number of variables \cite{belova18}. Though some of new reduction rules appear to be very useful in this parameterization, most of them increase the length of the formula, and hence cannot be used for solving the problem in this parameterization. In our work, we introduce new reduction rules that do not increase the formula length. Then, we decrease the parameter value to discount the 3-variables, given that $(n,3)$-MAXSAT can be solved in much better time than general MAXSAT. The combination of two techniques produces an algorithm with running time $O^∗(1.093^{|F|})$, improving the previous bound of $O^∗(1.106^{|F|})$ by Bansal and Raman \cite{bansal99}. 

\vspace{14pt}

\textit{Keywords:} Maximum Satisfiability, Parameterized Complexity, Exact Exponential Algorithms
\end{otherlanguage}

% vim: spell spelllang=ru_yo,en_gb


 \newgeometry{top=2cm,bottom=2cm,left=3cm,right=1.5cm,nohead,includeheadfoot}

\section*{Введение}
\label{sec:intro}
\addcontentsline{toc}{section}{\nameref{sec:intro}}

\subsection*{Актуальность задачи}

\firstpar{}Задача о максимальной разрешимости, или, сокращённо, MAXSAT, как оптимизационная версия задачи о разрешимости (сокращённо SAT), возможно, одной из самых известных NP-полных задач, имеет широкий круг применений, от анализа данных \cite{berg2015applications} до медицины \cite{lin2012application}.
При этом не только задача MAXSAT, но многие её частные случаи, такие, как $(n,3)$-MAXSAT, являются NP-трудными \cite{raman1998simplified}.

Гипотеза об экспоненциальном времени говорит, что задача 3SAT, то есть задача булевой разрешимости с дополнительным ограничением, что длина каждого дизъюнкта не более трёх, не может быть решена быстрее, чем за экспоненциальное время от количества переменных и тем более от длины входа. Как следствие, задача о максимальной разрешимости также не может быть решена за субэкспоненциальное время от длины входа, так как наличие такого алгоритма автоматически влекло бы за собой существование алгоритма для 3-SAT. Поэтому основное направление исследований в этой области -- уменьшение основания экспоненты.

В последние годы активно продвинулось изучение других параметризаций той же задачи: параметризация количеством выполненных дизъюнктов \cite{chen15} и общим количеством дизъюнктов \cite{xu19}. Для формул с большой средней длиной дизъюнкта эти алгоритмы дают хорошее время работы. Однако если в формуле большинство дизъюнктов имеют длину 1 или 2, алгоритм относительно длины применять эффективнее. При этом стоит отметить, что задача MAX-2-SAT, где все дизъюнкты имеют длину 1 или 2, уже является NP-трудной, хотя для неё существуют специальные алгоритмы, позволяющие решать её быстрее, чем в общем случае \cite{golovnev2014new}. Тем не менее, если ограничения на максимальную длину дизъюнкта нет, но средняя длина небольшая, задача эффективно решается именно алгоритмом относительно длины формулы.

\subsection*{Условие задачи}

\firstpar{}Задача MAXSAT формулируется следующим образом:

\begin{center}
 \begin{tabular}{|lp{.8\textwidth}|}
  \hline
  \multicolumn{2}{|l|}{MAXSAT} \\
  \textbf{Вход:} & Булева формула $F$ в конъюнктивной нормальной форме (КНФ) и число $k$ \\
  \textbf{Ответ:} & Означивание переменных, выполняющее хотя бы $k$ дизъюнктов. \\
  \hline
 \end{tabular}
\end{center}

Длина формулы обозначается за $L$.

Как упоминалось выше, цель работы -- создать алгоритм за $O^*(\alpha^L)$ при минимальном $\alpha$.
Алгоритм будет иметь следующую структуру:

\begin{enumerate}
 \item Свести экземпляр задачи к экземпляру задачи $(n,5)$-MAXSAT.
 
 \item Свести экземпляр задачи к экземпляру задачи $(n,3)$-MAXSAT.

 \item Запустить на полученном экземпляре лучший известный алгоритм для $(n,3)$-MAXSAT \cite{belova18}.
\end{enumerate}

Данная структура формирует представление о параметризациях, используемых на разных этапах алгоритма. На первом этапе мы решаем задачу относительно длины формулы. На втором этапе мы используем новый специально введённый нами параметр, уменьшающий стоимость 3-переменных, позволяющий использовать замечательное время работы алгоритма для $(n,3)$-MAXSAT. Наконец, на третьем этапе, мы используем естественную для $(n,3)$-MAXSAT параметризацию количеством переменных.

\subsection*{Ограничения работы}

\firstpar{}Алгоритм для задачи булевой разрешимости (SAT), работающий за $O^*(1.074^L)$, был представлен Гиршем в 2000 году \cite{hirsch2000new}. Несмотря на то, что целью работы является подойти ближе к этой границе, едва ли удастся её преодолеть, так как существование лучшего алгоритма для MAXSAT повлекло бы существование лучшего алгоритма для более простой задачи SAT. В частности, все худшие случаи представленного алгоритма в задаче булевой разрешимости разбирались бы тривиально.

В данной работе рассматривается лишь алгоритм относительно длины формулы, но не другие варианты параметризации задачи MAXSAT.

\subsection*{Определения ключевых терминов}

Булевы переменные в работе обозначаются буквами $x$, $y$, $z$, $w$.

Если $x$ — булева переменная, то выражения $x$ и $\ovl{x}$ называются литералами. Во избежание неоднозначности литералы в работе обозначаются буквами $l$, $k$, $m$.

Дизъюнкт — это дизъюнкция литералов, то есть выражение вида $x_1 \vee \ovl{x_2} \vee x_3 \vee \ldots$. По умолчанию считается, что повторяющихся литералов в дизъюнкте нет, иначе их можно было бы сократить по правилу $l \vee l = l$. В работе дизъюнкты обозначаются буквами $C$, $D$, $E$.

Формула находится в конъюнктивной нормальной форме (КНФ), если она является конъюнкцией дизъюнктов (то есть имеет вид $(x_1 \vee \ovl{x_2} \vee x_3) \wedge \ovl x_1 \wedge \ldots$).

Задача булевой разрешимости состоит в том, чтобы определить, существует ли означивание переменных, выполняющее формулу в КНФ. У неё есть вариант $k$-SAT с дополнительным ограничением на длину каждого дизъюнкта: она не больше $k$. В то время как задача 3SAT уже является NP-трудной, для задачи 2SAT известно полиномиальное решение.

Задача о максимальной разрешимости, как сформулировано выше, является оптимизационной версией этой задачи, и требует выполнения не всех, но хотя бы заданного числа дизъюнктов. У неё есть аналогичные варианты MAX-$k$-SAT. Кроме того, выделяют версии этой задачи $(n,k)$-MAXSAT, в которых каждая переменная входит в формулу не более $k$ раз. При этом задачи MAX-2-SAT, $(n,3)$-MAXSAT и даже $(n,3)$-MAX-2-SAT уже являются NP-трудными \cite{raman1998simplified}. Для задачи $(n,2)$-MAXSAT существует полиномиальное решение.

Переменная называется $k$-переменной, если она входит в формулу ровно $k$ раз.
Если про переменную известно, что она входит в формулу хотя бы $k$ раз, она называется $k^+$-переменной. Аналогично, если известно, что переменная входит в формулу не более $k$ раз, она называется $k^-$-переменной. Число $k$-переменных в формуле обозначается за $n_k$.

Если переменная $x$ входит $k$ раз положительно (то есть как литерал $x$) и $l$ раз отрицательно (как литерал $\ovl{x}$), она называется $(k,l)$-переменной. Такое же обозначение вводится и для литералов. Поскольку замена переменной $x$ на $\ovl{x}$ во всей формуле не влияет на ответ на задачу, если не указано иного, при обозначении переменных всегда считается $k \geq l$.

Алгоритм состоит из правил сокращения и правил расшепления.

Правило сокращения — полиномиальный алгоритм, преобразующий экземпляр задачи в эквивалентный ему и при этом не увеличивающий длину формулы. Такие правила применяются к формуле постоянно, пока это возможно. В правилах расщепления считается, что ни одно правило сокращения к формуле неприменимо.

Правило расщепления — полиномиальный алгоритм, преобразующий экземпляр задачи в несколько других вариантов и при этом с необходимостью уменьшающий длину формулы в каждом из них. От каждого из них алгоритм запускается рекурсивно, если хотя бы в одном из них удаётся получить положительный ответ, ответ на задачу является положительным.

Если во вариантах, произведённых правилом расщепления, длина уменьшается на $a_1, \dots, a_k$, то время работы алгоритма оценивается как рекуррентное соотношение

$$
 T(L) = T(L - a_1) + \dots + T(L - a_k)
$$

Решением такого соотношения является $T(n) = O^*(\alpha^n)$, где $\alpha$ -- единственный больший единицы корень уравнения

$$
 1 = \alpha^{-a_1} + \dots + \alpha^{-a_k}
$$

Вектор $(a_1, \dots, a_k)$ называется вектором расщепления, а число $\alpha$ -- числом расщепления. Асимптотика всего алгоритма оценивается как $O^*(\alpha^L)$, где $\alpha$ -- максимальное по всем правилам расщепления число расщепления.

Расщепление по переменной $x$ -- правило расщепления, разделяющееся на случаи $x = 0$ и $x = 1$. Разумеется, такое правило исчерпывает все варианты и по ответам в каждом из вариантов можно восстановить ответ на исходную формулу.

Подформула -- это подмножество дизъюнктов исходной формулы. Подформула называется замкнутой, если все переменные, литералы которых содержатся в подформуле, не имеют литералов вне этой подформулы.

\subsection*{Полученные результаты}

\firstpar{}Верхняя оценка на задачи максимальной разрешимости улучшена в данной работе с $O^*(1.106^L)$ до $O^*(1.093^L)$. В логарифмической шкале это даёт улучшение с $O^*(2^{0.145L})$ до $O^*(2^{0.128L})$, то есть на 11.7\%.

Кроме того, в работе предложена новая параметризация для задачи MAXSAT с уменьшенной стоимостью 3-переменных. Продолжение исследований в этом направлении может помочь сдвинуть эту границу ещё дальше.

% vim: spell spelllang=ru_yo,en_gb


 \newgeometry{top=2cm,bottom=2cm,left=3cm,right=1.5cm,nohead,includeheadfoot}

\section{Обзор литературы}
\label{sec:literature-review}

\subsection{История развития области}
\label{subsec:literature-review:history}

\firstpar{}Задача булевой разрешимости была исторически первой задачей, для которой была доказана NP-полнота (этот известный факт называется теоремой Кука-Левина). Задача о максимальной разрешимости, как оптимизационная версия этой задачи, автоматически является NP-полной и изучается с тех времён.

История развития точных алгоритмов для MAXSAT представлена в таблице \ref{table:maxsat-length-research}.

\begin{table}[ht]
 \caption{История развития алгоритмов для задачи MAXSAT}
 \centering
 \begin{tabular}{|c|c|c|c|}
  \hline
  \textbf{Работа} & \textbf{Год} & \textbf{Результат} & \textbf{Δ} \\
  \hline
  Нидермайер и Россманит \cite{niedermeier1999new} & \citeyear{niedermeier1999new} & $O^*(1.1279^L) = O^*(2^{0.1737L})$ & -- \\
  Банзал и Раман \cite{bansal99} & \citeyear{bansal99} & $O^*(1.1057^L) = O^*(2^{0.1450L})$ & 16.5\% \\
  \hline
 \end{tabular}
 \label{table:maxsat-length-research}
\end{table}

Видно, что со времён работы \cite{bansal99} значимого улучшения не произошло. Причиной тому является недостаточное развитие смежного алгоритма для $(n,3)$-MAXSAT.

Как будет продемонстрированно в данном обзоре чуть ниже, простые правила сокращения позволяют убрать из формулы 1- и 2-перменные, таким образом. можно считать, что все переменные в $(n,3)$-MAXSAT-формуле являются 3-переменными, а длина формулы, таким образом, равняется утроенному количеству переменных. Одним из направлений исследований алгоритмов для задачи $(n,3)$-MAXSAT является исследование алгоритмов, экспоненциальных относительно количества переменных в формуле. В силу равненства $L = 3n$ таковой алгоритм автоматически является и алгоритмом относительно длины. История этих алгоритмов приведена в таблице \label{table:n3-maxsat-research}.

\begin{table}[ht]
 \caption{История развития алгоритмов для задачи $(n,3)$-MAXSAT}
 \centering
 \begin{tabular}{|c|c|c|}
  \hline
  \textbf{Работа} & \textbf{Год} & \textbf{Результат} \\
  \hline
  Раман, Равикумар и Рао \cite{raman1998simplified} & \citeyear{raman1998simplified} & $O^*(1.732^n)$ \\
  Банзал и Раман \cite{bansal99} & \citeyear{bansal99} & $O^*(1.3248^n)$ \\
  Куликов и Куцков \cite{kulikov2009new} & \citeyear{kulikov2009new} & $O^*(1.2721^n)$ \\
  Близнец \cite{bliznets2013new} & \citeyear{bliznets2013new} & $O^*(1.2600^n)$ \\
  Чэнь, Сюй и Ван \cite{chen15} & \citeyear{chen15} & $O^*(1.237^n)$ \\
  Ли, Cюй, Ван и Ян \cite{li2017improved} & \citeyear{li2017improved} & $O^*(1.194^n)$ \\
  Белова и Близнец \cite{belova18} & \citeyear{belova18} & $O^*(1.191^n)$ \\
  \hline
 \end{tabular}
 \label{table:n3-maxsat-research}
\end{table}

Отметим, что для достижения заявленной нами асимптотики с помощью введённого нами параметра, необходимо существование алгоритма для $(n,3)$-MAXSAT, работающего не хуже, чем за время $O^*(1.194^n)$.

В свою очередь, активное развитие алгоритмов для $(n,3)$-MAXSAT стало возможным благодаря представленной Близнецом и Головнёвым \cite{bliznets12} идеи сведения задачи к задаче о покрытии множества. Эта идея первоначально была применена к задаче о максимальной разрешимости, параметризованной ответом. История работ, связанных с этой формулировкой задачи, приведены в таблице \ref{table:maxsat-answer-research}.

\begin{table}[ht]
 \caption{Развитие алгоритмов для MAXSAT относительно ответа}
 \centering
 \begin{tabular}{|c|c|c|c|}
  \hline
  \textbf{Работа} & \textbf{Год} & \textbf{Результат} & \textbf{Δ} \\
  \hline
  Махаджан и Раман \cite{mahajan1999parameterizing} & \citeyear{mahajan1999parameterizing} & $O^*(1.618^k) = O^*(2^{0.695k})$ & -- \\
  Нидермайер и Россманит \cite{niedermeier1999new} & \citeyear{niedermeier1999new} & $O^*(1.400^k) = O^*(2^{0.486k})$ & 30\% \\
  Банзал и Раман \cite{bansal99} & \citeyear{bansal99} & $O^*(1.381^k) = O^*(2^{0.466k})$ & 4\% \\
  Чэнь и Кандж \cite{chen2004improved} & \citeyear{chen2004improved} & $O^*(1.370^k) = O^*(2^{0.455k})$ & 2.5\% \\
  Близнец и Головнёв \cite{bliznets12} & \citeyear{bliznets12} & $O^*(1.358^k) = O^*(2^{0.442k})$ & 2.8\% \\
  Чень и Сунь \cite{chen15} & \citeyear{chen15} & $O^*(1.325^k) = O^*(2^{0.406^k})$ & 8\% \\
  \hline
 \end{tabular}
 \label{table:maxsat-answer-research}
\end{table}

Отдельно хочется отметить развитие алгоритмов для задачи о максимальной разрешимости, параметризованной общим количеством дизъюнктов. Лучший результат в этой области $O^*(1.2989^m)$ был получен Сунь и др. \cite{xu19} в \citeyear{xu19} году. В то время как эта работа также использует идею сведения к задаче о покрытии множества, там также введено большое количество новых правил редукции. В то время как большая их часть увеличивает длину формулы (и, следовательно, не может быть применена к рассматриваемой задаче), одно из них оказывается весьма полезным. Подробнее об этом рассказано в данном обзоре ниже.


 \begin{otherlanguage}{english}
 \printbibliography[heading=bibintoc,title={Список литературы}]
\end{otherlanguage}

% vim: spell spelllang=ru_yo,en_gb

\end{document}

% vim: spell spelllang=ru_yo,en_gb
